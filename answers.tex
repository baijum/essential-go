\cleardoublepage
\phantomsection
\chapter*{Appendix A: Answers}
\addcontentsline{toc}{chapter}{Appendix A: Answers}
\markboth{Appendix A: Answers}{Appendix A: Answers}

\section*{Chapter 2: Quick Start}

\textbf{Problem 1:} Write a function to check whether the first letter
in a given string is capital letters in English (A,B,C,D etc).

\textbf{Solution:}

\lstinputlisting[numbers=none]{code/answers/quick-start/capital.go}

\textbf{Problem 2:} Write a function to generate Fibonacci numbers
below a given value.

\textbf{Solution:}

\lstinputlisting[numbers=none]{code/answers/quick-start/fibonacci.go}

\section*{Chapter 3: Control Structures}

\textbf{Problem 1:} Write a program to print greetings based on time.
Possible greetings are Good morning, Good afternoon and Good evening.

\textbf{Solution:}

\lstinputlisting[numbers=none]{code/answers/control-structures/greetings.go}

\textbf{Problem 2:} Write a program to check if the given number is divisible by 2, 3, or 5.

\textbf{Solution:}

\lstinputlisting[numbers=none]{code/answers/control-structures/multiple.go}

\section*{Chapter 4: Data Structures}

\textbf{Problem 1:} Write a program to record temperatures for different locations and check if it's freezing for a given place.

\textbf{Solution:}

\lstinputlisting[numbers=none]{code/answers/data-structures/temperature.go}

\textbf{Problem 2:} Create a map of world nations and details. The key could
be the country name and value could be an object with details including capital,
currency, and population.

\textbf{Solution:}

\lstinputlisting[numbers=none]{code/answers/data-structures/nations.go}

\section*{Chapter 5: Functions \& Methods}

\textbf{Problem 1:} Write a program with a function to calculate the perimeter of a circle.

\textbf{Solution:}

\lstinputlisting[numbers=none]{code/answers/functions/circle.go}

\section*{Chapter 6: Interfaces}

\textbf{Problem 1:} Implement the built-in \texttt{error} interface for a custom data type.  This is how the \texttt{error} interface is defined:

\begin{lstlisting}[numbers=none]
type error interface {
    Error() string
}
\end{lstlisting}

\textbf{Solution:}

\lstinputlisting[numbers=none]{code/answers/interfaces/error.go}

\section*{Chapter 7: Concurrency}

{\bfseries Problem 1:} Write a program to watch log files and detect
any entry with a particular word.

\textbf{Solution:}

\lstinputlisting[numbers=none]{code/answers/concurrency/watchlog.go}

\section*{Chapter 8: Packages}

{\bf Problem 1:} Create a package with 3 source files and
another \textit{doc.go} for documentation.  The package should provide
functions to calculate areas for circle, rectangle, and triangle.

\textbf{Solution:}

circle.go:

\lstinputlisting[numbers=none]{code/answers/packages/docs/circle.go}

rectangle.go:

\lstinputlisting[numbers=none]{code/answers/packages/docs/rectangle.go}

triangle.go:

\lstinputlisting[numbers=none]{code/answers/packages/docs/triangle.go}

doc.go:

\lstinputlisting[numbers=none]{code/answers/packages/docs/doc.go}

\section*{Chapter 9: Input/Output}

\textbf{Problem 1:} Write a program to format a complex number as used in mathematics.  Example: \texttt{2 + 5i}

Use a struct like this to define the complex number:

\begin{lstlisting}[numbers=none]
type Complex struct {
    Real float64
    Imaginary float64
}
\end{lstlisting}

\textbf{Solution:}

\lstinputlisting[numbers=none]{code/answers/io/complex.go}

\section*{Chapter 10: Testing}

\textbf{Problem 1:} Write a test case program to fail the test and not continue with the remaining tests.

\textbf{Solution:}

\lstinputlisting[numbers=none]{code/answers/testing/failnow/failnow_test.go}

\section*{Chapter 11: Tooling}

{\bfseries Problem 1:} Write a program with exported type and methods
with documentation strings.  Then print the documentation
using the \texttt{go doc} command.

\textbf{Solution:}

Here is the package definition for a circle object:

\lstinputlisting[numbers=none]{code/answers/tooling/circle.go}

The docs can be accessed like this:

\begin{lstlisting}[numbers=none]
$ go doc
package circle // import "."

Package defines a circle object

type Circle struct{ ... }

$ go doc  Circle
type Circle struct {
        Radius float64
}
    Circle represents a circle shape


func (c Circle) Area() float64

$ go doc  Circle.Area
func (c Circle) Area() float64
    Area return the area of a circle
\end{lstlisting}
