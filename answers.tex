\cleardoublepage
\phantomsection
\chapter*{Appendix A: Answers}
\addcontentsline{toc}{chapter}{Appendix A: Answers}
\markboth{Appendix A: Answers}{Appendix A: Answers}

\section*{Chapter 2: Quick Start}

\textbf{Problem 1:} Write a function to check whether the first letter
in a given string is capital letters in English (A,B,C,D etc).

\textbf{Solution:}

\begin{lstlisting}[numbers=none]
package main

import "fmt"

func StartsCapital(s string) bool {
    for _, v := range "ABCDEFGHIJKLMNOPQRSTUVWXYZ" {
        if string(s[0]) == string(v) {
            return true
        }
    }
    return false
}

func main() {
    h := StartsCapital("Hello")
    fmt.Println(h)
    w := StartsCapital("world")
    fmt.Println(w)
}
\end{lstlisting}

\textbf{Problem 2:} Write a function to generate Fibonacci numbers
below a given value.

\textbf{Solution:}

\begin{lstlisting}[numbers=none]
package main

import "fmt"

func Fib(n int) {
    for i, j := 0, 1; i < n; i, j = i+j, i {
        fmt.Println(i)
    }

}

func main() {
    Fib(200)
}
\end{lstlisting}

\section*{Chapter 3: Control Structures}

\textbf{Problem 1:} Write a program to print greetings based on time.
Possible greetings are Good morning, Good afternoon and Good evening.

\textbf{Solution:}

\begin{lstlisting}[numbers=none]
package main

import (
    "fmt"
    "time"
)

func main() {
    t := time.Now()
    switch {
    case t.Hour() < 12:
        fmt.Println("Good morning!")
    case t.Hour() < 17:
        fmt.Println("Good afternoon.")
    default:
        fmt.Println("Good evening.")
    }
}
\end{lstlisting}

\textbf{Problem 2:} Write a program to check if a given number is a multiple of 2, 3, or 5.

\textbf{Solution:}

\lstinputlisting[numbers=none]{code/answers/control-structures/multiple.go}

\section*{Chapter 4: Data Structures}

\textbf{Problem 1:} Write a program to record temperatures
in different locations and functionality to check whether its freezing or not.

\textbf{Solution:}

\lstinputlisting[numbers=none]{code/answers/data-structures/temperature.go}

\textbf{Problem 2:} Create a map of world nations and details. They key could
be the country name and value could be an object with details including capital,
currency, and population.

\textbf{Solution:}

\lstinputlisting[numbers=none]{code/answers/data-structures/nations.go}

\section*{Chapter 5: Functions \& Methods}

\textbf{Problem 1:} Write a program with a function to calculate perimeter of a circle.

\textbf{Solution:}

\begin{lstlisting}[numbers=none]
package main

import "fmt"

type Circle struct {
    Radius float64
}

// Area return the area of a circle for the given radius
func (c Circle) Area() float64 {
    return 3.14 * c.Radius * c.Radius
}

func main() {
    c := Circle{5.0}
    fmt.Println(c.Area())
}
\end{lstlisting}

\section*{Chapter 6: Interfaces}

\textbf{Problem 1:} Implement the built-in \texttt{error} interface for a custom data type.  This is how the \texttt{error} interface is defined:

\begin{lstlisting}[numbers=none]
type error interface {
    Error() string
}
\end{lstlisting}

\textbf{Solution:}

\begin{lstlisting}[numbers=none]
package main

import "fmt"

type UnauthorizedError struct {
    UserID string
}

func (e UnauthorizedError) Error() string {
    return "User not authorised: " + e.UserID
}

func SomeAction() error {
    return UnauthorizedError{"jack"}
}

func main() {
    err := SomeAction()
    if err != nil {
        fmt.Println(err)
    }
}
\end{lstlisting}

\section*{Chapter 7: Concurrency}

{\bfseries Problem 1:} Write a program to watch log files and detect
any entry with a particular word.

\textbf{Solution:}

\begin{lstlisting}[numbers=none]
package main

import (
    "bufio"
    "fmt"
    "os"
    "os/signal"
    "strings"
    "time"
)

func watch(word, fp string) error {

    f, err := os.Open(fp)
    if err != nil {
        return err
    }
    r := bufio.NewReader(f)
    defer f.Close()
    for {
        line, err := r.ReadBytes('\n')
        if err != nil {
            if err.Error() == "EOF" {
                time.Sleep(2 * time.Second)
                continue
            }
            fmt.Printf("Error: %s\n%v\n", line, err)
        }
        if strings.Contains(string(line), word) {
            fmt.Printf("%s: Matched: %s\n", fp, line)
        }
        time.Sleep(2 * time.Second)
    }
}

func main() {
    word := os.Args[1]
    files := []string{}
    for _, f := range os.Args[2:len(os.Args)] {
        files = append(files, f)
        go watch(word, f)
    }
    sig := make(chan os.Signal, 1)
    done := make(chan bool)
    signal.Notify(sig, os.Interrupt)
    go func() {
        for _ = range sig {
            done <- true
        }
    }()
    <-done
}
\end{lstlisting}

\section*{Chapter 8: Packages}

{\bf Problem 1:} Create a package with 3 source files and
another \textit{doc.go} for documentation.  The package should provide
functions to calculate areas for circle, rectangle, and triangle.

\textbf{Solution:}

circle.go:

\begin{lstlisting}[numbers=none]
package shape

// Circle represents a circle shape
type Circle struct {
    Radius float64
}

// Area return the area of a circle
func (c Circle) Area() float64 {
    return 3.14 * c.Radius * c.Radius
}
\end{lstlisting}

rectangle.go:

\begin{lstlisting}[numbers=none]
package shape

// Rectangle represents a rectangle shape
type Rectangle struct {
    Length float64
    Width float64
}

// Area return the area of a rectangle
func (r Rectangle) Area() float64 {
    return r.Length * r.Width
}
\end{lstlisting}

triangle.go:

\begin{lstlisting}[numbers=none]
package shape

// Triangle represents a rectangle shape
type Triangle struct {
    Breadth float64
    Height float64
}

// Area return the area of a triangle
func (t Triangle) Area() float64 {
    return (t.Breadth * t.Height)/2
}
\end{lstlisting}

doc.go:

\begin{lstlisting}[numbers=none]
// Package shape provides areas for different shapes
// This includes circle, rectangle, and triangle.
\end{lstlisting}

\section*{Chapter 9: Input/Output}

\textbf{Problem 1:} Write a program to format a complex number as used in mathematics.  Example: \texttt{2 + 5i}

Use a struct like this to define the complex number:

\begin{lstlisting}[numbers=none]
type Complex struct {
    Real float64
    Imaginary float64
}
\end{lstlisting}

\textbf{Solution:}

\lstinputlisting[numbers=none]{code/answers/io/complex.go}

\section*{Chapter 10: Testing}

\textbf{Problem 1:} Write a program to fail test and not continue with remaining tests.

\textbf{Solution:}

\lstinputlisting[numbers=none]{code/answers/testing/failnow/failnow_test.go}

\section*{Chapter 11: Tooling}

{\bfseries Problem 1:} Write a program with exported type and methods
with documentation strings.  Then print the documentation
using \texttt{go doc} command.

\textbf{Solution:}

Here is the package definition for a circle object:

\begin{lstlisting}[numbers=none]
// Package defines a circle object
package circle

// Circle represents a circle shape
type Circle struct {
    Radius float64
}

// Area return the area of a circle
func (c Circle) Area() float64 {
    return 3.14 * c.Radius * c.Radius
}
\end{lstlisting}

The docs can be accessed like this:

\begin{lstlisting}[numbers=none]
$ go doc
package circle // import "."

Package defines a circle object

type Circle struct{ ... }

$ go doc  Circle
type Circle struct {
        Radius float64
}
    Circle represents a circle shape


func (c Circle) Area() float64

$ go doc  Circle.Area
func (c Circle) Area() float64
    Area return the area of a circle
\end{lstlisting}
