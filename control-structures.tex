\cleardoublepage
\phantomsection
\chapter{Control Structures}

\begin{quote}
\textit{Science is what we understand well enough to explain to a
computer. Art is everything else we do.} --- Donald E. Knuth
\end{quote}

Control structure determines how the code block will get executed for
the given conditional expressions and parameters.  Go provides a
number of control structures
including \textit{for}, \textit{if}, \textit{switch}, \textit{defer},
and \textit{goto}.  The quickstart chapter has already introduced
control structures like \textit{if} and \textit{for}.  This chapter
will elaborate more about these topics and introduce some other
related topics.

\section{If}

\subsection{Basic If}

Programming involves lots of decision making based on the input
parameters.  The If\index{if} control structure allows to perform a
particular action on certain condition.  A conditional expressions is
what used for making decisions in the code using the If control
structure.  So, the If control structure will be always associated
with a conditional expression which evaluates to a boolean value.  If
the boolean value is true, the statements within the block will be
executed.  Consider this example:

\begin{lstlisting}[caption=If example program]
package main

import "fmt"

func main() {
    if 1 < 2 {
        fmt.Println("1 is less than 2")
     }
}
\end{lstlisting}

The first line starts with the \texttt{if} keyword followed by a
conditional expression and the line ends with an opening curly
bracket.  If the conditional expression is getting evaluated as true,
the statements given within the curly brace will get evaluated.  In
the above example, the conditional expression \texttt{1 < 2} will be
evaluated as true so the print statement given below that will get
executed.  In fact, you can add any number of statements within the
braces.

\subsection{If and Else}

Sometimes you need to perform different set of action if the condition
is not true.  Go provides a variation of the \texttt{if} syntax with
the \texttt{else}\index{if!else} block for that.

Consider this example with else block:

\begin{lstlisting}[caption=If with else block]
package main

import "fmt"

func main() {
    if 3 > 4 {
        fmt.Println("3 is greater than 4")
    } else {
        fmt.Println("3 is not greater than 4")
    }
}
\end{lstlisting}

In the above code, the code will be evaluated as false.  So, the
statements within \texttt{else} block will get executed.

\subsection{Example}

Now we will go through a complete example, the task is to identify the
given number is even or odd.  The input can be given as a command line
argument.

\begin{lstlisting}[caption=If with else example]
package main

import (
     "fmt"
     "os"
     "strconv"
)

func main() {
     i := os.Args[1]
     n, err := strconv.Atoi(i)
     if err != nil {
             fmt.Println("Not a number:", i)
             os.Exit(1)
     }
     if n%2 == 0 {
             fmt.Printf("%d is even\n", n)
     } else {
             fmt.Printf("%d is odd\n", n)
     }
}
\end{lstlisting}

The \texttt{os} package has an attribute named \texttt{Args}.  The
value of \texttt{Args} will be a slice of strings which contains all
command line arguments passed while running the program.  As we have
learned from the Quickstart chapter, the values can be accessed using
the index syntax.  The value at zero index will be the program name
itself and the value at 1st index the first argument and the value at
2nd index the second argument and so on.  Since we are expecting only
one argument, you can access it using the 1st index
(\texttt{os.Args[1]}).

The \texttt{strconv} package provides a function named \texttt{Atoi}
to convert strings to integers.  This function return two values, the
first one is the integer value and the second one is the error value.
If there is no error during convertion, the error value will
be \texttt{nil}.  If it's a non-nil value, that indicates there is an
error occured during conversion.

The \texttt{nil} is an identifier in Go which represents the ``zero
value'' for certain built-in types.  The \texttt{nil} is used as the
zero for these types: interfaces, functions, pointers, maps, slices,
and channels.

In the above example, the second expected value is an object
conforming to built-in \texttt{error} interface.  We will discuss more
about errors and interfaces in later chapters.  The zero value for
interface, that is \texttt{nil} is considered as there is no error.

The \texttt{Exit} function within \texttt{os} package helps to exit
the program prematurely.  The argument passed will be exit status
code.  Normally exit code \texttt{0} is treated as success and
non-zero value as error.

The conditional expression use the modulus operator to get remainder
and checking it is zero.  If the remainder against 2 is zero, the
value will be even otherwise the value will odd.

\subsection{Else If}

There is a third alternative syntax available for the If control
structure, that is \texttt{else if} block.  The Else If block get
executed if the conditional expression gives true value and previous
conditions are false.  It is possible to add any number of Else If
blocks based on the requirements.

Look at this example where we have three choices based on the age
group.

\begin{lstlisting}[caption=if example program output]
package main

import "fmt"

func main() {
        age := 10
        if age < 10 {
        fmt.Println("Junior", age)
    } else if age < 20 {
        fmt.Println("Senior", age)
    } else {
        fmt.Println("Other", age)
    }
}
\end{lstlisting}

In the above example, the value printed will be
either \texttt{Junior}, \texttt{Senior} or \texttt{Other}.  You can
change age value and run the program again and again to see the
outputs.  The Else If can be repeated here to create more choices.

\subsection{Inline Statement}

In the previous section, the variable \textit{age} was only within the
If, Else If and Else blocks.  And that variable was not used used
afterwards in the function.  Go provides a syntax to define a variable
along with the If where the scope of that variable will be within the
blocks.  In fact, the syntax can be used for any valid Go statement.
However, it is mostly used for declaring variables.\index{if!inline
statement}

Here is an example where a variable named \texttt{money} is declared
along with the If control structure.

\begin{lstlisting}[caption=If with initialization statement]
package main

import "fmt"

func main() {
    if money := 20000; money > 15000 {
        fmt.Println("I am going to buy a car.", money)
    } else {
        fmt.Println("I am going to buy a bike.", money)
    }
    // can't use the variable `money` here
}
\end{lstlisting}

As mentioned above, the variable declared by the inline statement is
available only within the scope of If, Else If and Else blocks.  So,
the variable \texttt{money} cannot be used outside the blocks.

It is possible to make any valid Go statement as part of the If
control structure.  For example, it is possible to call a function
like this:

\begin{lstlisting}[caption=Variable initialization in If]
    if money := someFunction(); money > 15000 {
\end{lstlisting}

\section{For}

\subsection{Basic For}

As we have seen briefly in the Quickstart, the For\index{for} control
structure helps to create loops to repeat certain actions.  The For
control structure has few syntax variants.

Consider a program to print few names.

\begin{lstlisting}[caption=For loop example (forbasic.go)]
package main

import "fmt"

func main() {
    names := []string{"Tom", "Polly", "Huck", "Becky"}
    for i := 0; i < len(names); i++  {
        fmt.Println(names[i])
    }
}
\end{lstlisting}

You can save the above program in a file named \texttt{names.go}
and run it like this:

\begin{lstlisting}[numbers=none]
$ go run name.go
Tom
Polly
Huck
Becky
\end{lstlisting}

In the above example, \texttt{names} variable hold a slice of strings.
The value of \texttt{i} is initialized to zero and incremented one by
one.  The \texttt{i++} statement increment the value of \texttt{i}.
The second part of \texttt{for} loop check if value of \texttt{i} is
less than length of the slice.  The built-in \texttt{len} gives the
length of slice.

Other programming languages offer many ways for iterations. Some of
the examples are \textit{while} and \textit{do...while}.  But in Go
using syntactic variation of \textit{for} loop meets all requirements.
Functional languages prefer to use recursion instead of iteration.

\subsection{Break Loop Prematurely}

Sometimes the iteration should be stopped prematurely on certain
condition.  This can be achieved using the If control structure and
break statement.  We have already studied If control structure from
the previous major section. The \texttt{break} keyword allows to
create a break statement.  The break\index{break!loop} statement end the
loop immediately.  Though any other code followed by For loop will be
executed.

Let's alter the previous program to stop printing after the
name \texttt{Polly} found.

\begin{lstlisting}[caption=For loop with break]
package main

import "fmt"

func main() {
    names := []string{"Tom", "Polly", "Huck", "Becky"}
    for i := 0; i < len(names); i++  {
        fmt.Println(names[i])
        if names[i] == "Polly" {
            break
        }
    }
}
\end{lstlisting}

In the above example, we added an If control structure to check for
the value of name during each iteration.  If the value
matches \texttt{Polly}, break statement will be executed.  The break
statement makes the For loop to end immediately.

As you can see in the above code, the break statement can stand alone
without any other input.  There is alternate syntax with
label\index{break!label} similar to how \textit{goto} works, which we
are going to see below.  This is useful when you have multiple loops
and want to break a particular one, may be the outer loop.

To understand this better, let's consider an example.  The problem is
to to change print the name given the slice until a word
with \texttt{u} found.

\begin{lstlisting}[caption=For loop with break and label]
package main

import "fmt"

func main() {
    names := []string{"Tom", "Polly", "Huck", "Becky"}
Outer:
    for i := 0; i < len(names); i++ {
        for j := 0; j < len(names[i]); j++ {
            if names[i][j] == 'u' {
                break Outer
            }
        }
        fmt.Println(names[i])
    }
}
\end{lstlisting}

In the above example, we are declaring a label statement just before
the first For loop.  There is an inner loop to iterate through the
name string and check for the presence of character \texttt{u}.  If
the character \texttt{u} is found, then it will break the outer loop.
If the label \texttt{Outer} is not used in the break statement, then
the inner loop will be stopped.

\subsection{Partially Execute Loop Statements}

Sometimes statements within For loop should be executed on certain
iterations.  Go has a \texttt{continue}\index{continue} statement to
proceed loop without executing further statements.

Let's modify the previous problem to print all names
except \texttt{Polly}.

\begin{lstlisting}[caption=For loop with continue]
package main

import "fmt"

func main() {
    names := []string{"Tom", "Polly", "Huck", "Becky"}
    for i := 0; i < len(names); i++ {
        if names[i] == "Polly" {
            continue
        }
        fmt.Println(names[i])
    }
}
\end{lstlisting}

In the above code, the \texttt{continue} statement makes it proceed
with next iteration in the loop without printing \texttt{Polly}.

Similar to \texttt{break} statement with label, continue also can be
used with a label\index{continue!label}.  This is useful if there are
multiple loops and want to continue a particular loop, say the outer
one.

Let's consider an example where you need to print names which doesn't
have character \texttt{u} in it.

\begin{lstlisting}[caption=For loop with continue and label]
package main

import "fmt"

func main() {
    names := []string{"Tom", "Polly", "Huck", "Becky"}
Outer:
    for i := 0; i < len(names); i++ {
        for j := 0; j < len(names[i]); j++ {
            if names[i][j] == 'u' {
                continue Outer
            }
        }
        fmt.Println(names[i])
    }
}
\end{lstlisting}

In the above code, just before the first loop a label is declared.
Later inside the inner loop to iterate through the name string and
check for the presence of character \texttt{u}.  If the
character \texttt{u} is found, then it will continue the outer loop.
If the label \texttt{Outer} is not used in the continue statement,
then the inner loop will be proceed to execute.

\subsection{For with Outside Initialization}

The statement for value initialization and the last pat to increment
value can be removed from the For control structure.  The value
initialization can be moved outside For and value increment can be
moved inside loop.

The previous example can be changed like this:

\begin{lstlisting}[caption=For without initialization and increment]
package main

import "fmt"

func main() {
    names := []string{"Tom", "Polly", "Huck", "Becky"}
    i := 0
    for i < len(names) {
        i++
        fmt.Println(names[i])
    }
}
\end{lstlisting}

In the above example, the scope of variable \texttt{i} is outside For
loop code block.  Whereas in the previous section, when the variable
declared along with For loop, the scope of that variable was within
the loop code block.

\subsection{Infinite Loop}

For loop has yet another syntax variant to support infinite
loop\index{for!infinite}.  You can create a loop that never ends until
explicitly stopped using break or exiting the whole program.  To
create an infinite loop, you can use the \texttt{for} keyword followed
by the curly bracket.

If any variable initialization is required, that should be declared
outside the loop.  Conditions can be added inside the loop.

The previous example can be changed like this:

\begin{lstlisting}[caption=Infinite For loop]
package main

import "fmt"

func main() {
    names := []string{"Tom", "Polly", "Huck", "Becky"}
    i := 0
    for {
        if i >= len(names) {
            break
        }
        fmt.Println(names[i])
        i++
    }
}
\end{lstlisting}

\subsection{Range Loops}

The range\index{range!loop} clause form of the for loop iterates over
a slice or map.  When looping over a slice using range, two values are
returned for each iteration.  The first is the index, and the second
is a copy of the element at that index.

The previous example \texttt{for} loop can be simplified using
the \texttt{range} clause like this:

\begin{lstlisting}[caption=Range loop with slice]
package main

import "fmt"

func main() {
    characters := []string{"Tom", "Polly", "Huck", "Becky"}
    for _, j := range characters {
        fmt.Println(j)
    }
}
\end{lstlisting}

The underscore is called blank indentifier, the value assigned to that
variable will be ignored.  In the above example, the index values will
be assigned to the underscore.

The range loop can be used with map.  Here is an example:

\begin{lstlisting}[caption=Range loop with map]
package main

import "fmt"

func main() {
    var characters = map[string]int{
                "Tom": 8,
                "Polly": 51,
                "Huck": 9,
                "Becky": 8,
    }
    for name, age := range characters {
        fmt.Println(name, age)
    }
}
\end{lstlisting}

\section{Switch Cases}

\subsection{Basic Switch}

In addition to the \texttt{if} condition, Go provides \texttt{switch
case}\index{switch} control structure for branch instructions.
The \texttt{switch case} is more convenient if many cases need to be
handled in the branch instructions.

The below program use a switch case to print number names based on the
value.

\begin{lstlisting}[caption=Switch case example]
package main

import "fmt"

func main() {
    v := 1
    switch v {
    case 0:
            fmt.Println("zero")
    case 1:
            fmt.Println("one")
    case 2:
            fmt.Println("two")
    default:
            fmt.Println("unknown")
    }
}
\end{lstlisting}

In this case, the value of \texttt{v} is \texttt{1}, so the case that
is going to execute is 2nd one.  This will be the output.

\begin{lstlisting}[numbers=none]
$ go run switchbasic.go
one
\end{lstlisting}

If you change the value of \texttt{v} to \texttt{0}, it's going to
print \texttt{zero} and for \texttt{2} it will print \texttt{two}.  If
the value is any number other than \texttt{0}, \texttt{1}
or \texttt{2}, it's going to print \texttt{unknown}.

\subsection{Fallthrough}

The cases are evaluated top to bottom until a match is found.  If a
case is matched, the statements within that case will be executed. And
no other case will be executed unless
a \texttt{fallthrough}\index{fallthrough} statement is used.
The \texttt{fallthrough} must be the last statement within the case.

Here is a modified version with \texttt{fallthrough}

\begin{lstlisting}[caption=Switch case with fallthrough]
package main

import "fmt"

func main() {
    v := 1
    switch v {
    case 0:
            fmt.Println("zero")
    case 1:
            fmt.Println("one")
            fallthrough
    case 2:
            fmt.Println("two")
    default:
            fmt.Println("unknown")
    }
}
\end{lstlisting}

If you run this program, it will print \texttt{one} followed
by \texttt{two}.

\begin{lstlisting}[numbers=none]
$ go run switchbasic.go
one
two
\end{lstlisting}

\subsection{Break}

As you can see from the above examples, the switch statements
break\index{break!switch} implicitly at the end of each cases.
The \texttt{fallthrough} statement can be used to passdown control to
the next case.  However, sometimes execution should be stopped early
without executing all statements.  This can can be achieved
using \texttt{break} statements.

Here is an example:

\begin{lstlisting}[caption=Switch case with break]
package main

import (
    "fmt"
    "time"
)

func main() {
    v := "Becky"
    t := time.Now()
    switch v {
    case "Huck":
        if t.Hour() < 12 {
            fmt.Println("Good morning,", v)
            break
        }
        fmt.Println("Hello,", v)
    case "Becky":
        if t.Hour() < 12 {
            fmt.Println("Good morning,", v)
            break
        }
        fmt.Println("Hello,", v)
    default:
        fmt.Println("Hello")
    }
}
\end{lstlisting}

In the above example, morning time greeting is different.

\subsection{Multiple Cases}

Multple cases\index{switch!multiple cases} can be presented in
comma-separated lists.

Here is an example.

\lstinputlisting[caption=Switch with multiple cases]{code/control-structures/switchmultiple.go}

In this example, if any of the value is matched in the given list,
that case will be executed.

\subsection{Without Expression}

If the switch has no expression\index{switch!without expression} it
switches on true.  This is useful to write an if-else-if-else chain.

Let's take the example program used earlier when Else If was
introduced:

\lstinputlisting[caption=Switch without expression]{code/control-structures/switchnoexpression.go}

\section{Defer Statements}

Sometimes it will require to force certain things to do before a
function returns.  For example, closing an opened file descriptor.  Go
provides the \textit{defer}\index{defer} statements to do these kind
of cleanup actions.

A defer statement add a function call into a stack. The stack of
function call executes at the end of the surrounding function in a
last-in-first-out (LIFO) order. Defer is commonly used to perform
various clean-up actions.

Here is a simple example:

\begin{lstlisting}[caption=Defer usage]
package main

import (
    "fmt"
    "time"
)

func main() {
    defer fmt.Println("world")
    fmt.Println("hello")
}
\end{lstlisting}

The above program is going to print \texttt{hello} followed
by \texttt{world}.

If there are multiple \textit{defer} statements, it will execute in
last-in-first-out (LIFO) order.

Here is a simple example to demonstrate it:

\begin{lstlisting}[caption=Defer usage]
package main

import "fmt"

func main() {
    for i := 0; i < 5; i++ {
        defer fmt.Println(i)
    }
}
\end{lstlisting}

The above program will print this output:

\begin{lstlisting}[numbers=none]
4
3
2
1
0
\end{lstlisting}

The arguments passed the the deferred call are evaluated immediately.
But the deferred call itself is not executed until the function
returns.  Here is a simple example to demonstrate it:

\begin{lstlisting}[caption=Defer argument evaluation]
package main

import (
    "fmt"
    "time"
)

func main() {
    defer func(t time.Time) {
        fmt.Println(t, time.Now())
    }(time.Now())
}
\end{lstlisting}

When you run the above program, you can see a small difference in
time.  The \textit{defer} can also be used to recover
from \textit{panic}, which will be discussed in the next section.

\section{Deffered Panic Recover}

We have discussed the commonly used control structures including if,
for, and switch.  This section is going to discuss a less commonly
used set of control structures: \textit{defer}, \textit{panic},
and \textit{recover}.  We have discussed the use of the defer
statement in the previous section. In this section, you are going to
learn how to use the \textit{defer} along with \textit{panic}
and \textit{recover}.\index{panic}\index{recover}

Few important points about defer, panic, and recover:

\begin{itemize}
\item A panic causes the program stack to begin unwinding and recover can stop it
\item Deferred functions are still executed as the stack unwinds
\item If recover is called inside such a deferred function, the stack stops unwinding
\item The recover returns the value (as an \textit{interface\{\}}) that was passed to panic
\item A panic cannot be recovered by a different goroutine
\end{itemize}

Here is an example:

\lstinputlisting[caption=deferred panic recover]{code/control-structures/panic.go}

\section{Goto}

The \textit{goto}\index{goto} statement can be used to jump control to
another statement.  The location to where the control should be passed
is specified using label statements.  The \textit{goto} statement and
the corresponding label should be within the same function.
The \textit{goto} cannot jump to a label inside another code block.

\begin{lstlisting}[caption=Goto example program (goto.go)]
package main

import "fmt"

func main() {
    num := 10
    goto Marker
    num = 20
Marker:
    fmt.Println("Value of num:", num)
}
\end{lstlisting}

You can save the above program in a file named \texttt{goto1.go} and
run it like this:

\begin{lstlisting}[caption=Goto example program output]
$ go run goto1.go
Value of num: 10
\end{lstlisting}

In the above code \texttt{Marker:} is a label statement.  A label
statement is a valid identifier followed by a colon.  A label
statement will be target for \textit{goto}, \textit{break}
or \textit{continue} statement.  We will look at \textit{break}
and \textit{continue} statement when we study the For control
structure.

The \textit{goto} statement is writen using the \texttt{goto} keyword
followed by a valid label name.  In the above code, immediately after
the \textit{goto} statement, there is a statement to assign a
different value to \texttt{num}.  But that statement is never getting
executed as the \textit{goto} makes the program to jump to the label.

\section{Exercises}

\textbf{Exercise 1:} Print whether the number given as the command line
argument is even or odd.

\textbf{Solution:}

You can store the program with a file named \texttt{evenodd.go}.
Later you can compile this program and then you will get a binary
executable with name as \texttt{evenodd}.  You can execute this
program like this:

\begin{lstlisting}[numbers=none]
./evenodd 3
3 is odd
./evenodd 4
4 is even
\end{lstlisting}

In the above program, the 3 and 4 are the command line arguments.  The
command line arguments can be accessed from Go using the slice
available under \texttt{os} package.  The arguments will be available
with exported name as \texttt{Args} and individual items can be
accessed using the index.  The 0th index contains the program itself,
so it can be ignored.  To access the 1st command argument
use \texttt{os.Args[1]}.  The values will be of type string which can
be converted later.

\lstinputlisting[numbers=none]{code/control-structures/exercise1/evenodd.go}

\textbf{Exercise 2:} Write a program to print numbers below 20 which are multiples of 3 or 5.

\textbf{Solution:}

\lstinputlisting[numbers=none]{code/control-structures/exercise2/multiples.go}

\subsection{Additional Exercises}

Answers to these additional exercises are given in the Appendix A.

\textbf{Problem 1:} Write a program to print greetings based on time.
Possible greetings are Good morning, Good afternoon and Good evening.

\textbf{Problem 2:} Write a program to check if a given number is a multiple of 2, 3, or 5.

\section*{Summary}

This chapter introduced control structures available in Go except those related
to concurrency. The \textit{if} control structure was covered first,
then \textit{for} loop explained. The \textit{switch} cases was discussed later.
Then \textit{defer} statement and finally \textit{goto} control structure was
explained in detail. This chapter also briefly explained about accessing command
line arguments from the program.
