\cleardoublepage
\phantomsection
\chapter{Data Structures}

\begin{quote}
\textit{Bad programmers worry about the code.  Good programmers
worry about data structures and their relationships.} --- Linus
Torvalds
\end{quote}

In the last last few chapters we have seen some of the primitive data
data types.  We also introduced few other advanced data types without
going to the details.  In this chapter, we are going to look into more
data structures.

\section{Primitive Data Types}

Advanced data structures are built on top of primitive data types. This section
is going to cover the primitive data types in Go.

\subsection{Zero Value}

In the Quick Start chapter, you have learned various ways to declare a variable.
When you declare a variable using the \texttt{var}
\index{var keyword} statement without assigning a value, a default Zero value
\index{zero value} will be assigned for certain types. The Zero value is 0 for
integers and floats, empty string for strings, and false for boolean. To
demonstrate this, here is an example:

\begin{lstlisting}
package main

import "fmt"

func main() {
    var name string
    var age int
    var tall bool
    var weight float64
    fmt.Printf("%#v, %#v, %#v, %#v\n", name, age, tall, weight)
}
\end{lstlisting}

This is the output:

\begin{lstlisting}[numbers=none]
"", 0, false, 0
\end{lstlisting}

\subsection{Variable}

In the quick start chapter, we have discussed about variables and its
usage.  The variable declared outside the function (package level) can
access anywhere within the same package.

Here is an example:

\begin{lstlisting}
package main

import (
    "fmt"
)

var name string
var country string = "India"

func main() {
    name = "Jack"
    fmt.Println("Name:", name)
    fmt.Println("Country:", country)
}
\end{lstlisting}

In the above example, the \texttt{name} and \texttt{country} are two
package level variables.  As we have seen above the \texttt{name} gets
zero value, where as value for \texttt{country} variable is explicitly
initialized.

If the variable has been defined using the \texttt{:=} syntax, and the
user wants to change the value of that variable, they need to
use \texttt{=} instead of \texttt{:=} syntax.

If you run the below program, it's going to throw an error:

\begin{lstlisting}
package main

import (
    "fmt"
)

func main() {
    age := 25
    age := 35
    fmt.Println(age)
}
\end{lstlisting}

\begin{lstlisting}[numbers=none]
$ go run update.go
# command-line-arguments
./update.go:9:6: no new variables on left side of :=
\end{lstlisting}

The above can be fixed like this:

\begin{lstlisting}
package main

import (
    "fmt"
)

func main() {
    age := 25
    age = 35
    fmt.Println(age)
}
\end{lstlisting}

Now you should see the output:

\begin{lstlisting}[numbers=none]
$ go run update.go
35
\end{lstlisting}

Using the \textit{reflect} package\index{reflect}, you can identify
the type of a variable:

\begin{lstlisting}
package main

import (
       "fmt"
       "reflect"
)

func main() {
     var pi = 3.41
     fmt.Println("type:", reflect.TypeOf(pi))
}
\end{lstlisting}

Using one or two letter variable names inside a function is common
practice.  If the variable name is multi-word, use lower camelCase
(initial letter lower and subsequent words capitalized) for unexported
variables.  If the variable is an exported one, use upper CamelCase
(all the words capitalized).  If the variable name contains any
abbreviations like ID, use capital letters.  Here are few examples:
pi, w, r, ErrorCode, nodeToDaemonPods, DB, InstanceID.

\subsubsection{Unused variables and imports}

If you declare a variable inside a function, use that variable
somewhere in the same function where it is declared.  Otherwise, you
are going to get a compile error.  Whereas a global variable declared
but unused is not going to throw compile time error.

Any package that is getting imported should find a place to use.
Unused import also throws compile time error.

\subsection{Boolean Type}

A boolean type represents a pair of truth values.  The truth values
are denoted by the constants \textit{true} and \textit{false}.  These
are the three logical operators that can be used with boolean values:

\begin{itemize}
\item \texttt{\&\&} -- Logical AND
\item \texttt{||} -- Logical OR
\item \texttt{!} -- Logical NOT
\end{itemize}

Here is an example:

\lstinputlisting{code/data-structures/logical.go}

The output of the above logical operators are like this:

\begin{lstlisting}[numbers=none]
$ go run logical.go
false
true
false
true
\end{lstlisting}

\subsection{Numeric Types}

The numeric type includes both integer types and floating-point types.
The allowed values of numeric types are same across all the CPU
architectures.

These are the unsigned integers: 

\begin{itemize}
\item  uint8 -- the set of all unsigned  8-bit integers (0 to 255)
\item  uint16 -- the set of all unsigned 16-bit integers (0 to 65535)
\item  uint32 -- the set of all unsigned 32-bit integers (0 to 4294967295)
\item  uint64 -- the set of all unsigned 64-bit integers (0 to 18446744073709551615)
\end{itemize}

These are the signed integers:

\begin{itemize}
\item int8 -- the set of all signed  8-bit integers (-128 to 127)
\item int16 -- the set of all signed 16-bit integers (-32768 to 32767)
\item int32 -- the set of all signed 32-bit integers (-2147483648 to 2147483647)
\item int64 -- the set of all signed 64-bit integers (-9223372036854775808 to 9223372036854775807)
\end{itemize}

These are the two floating-point numbers:

\begin{itemize}
\item float32 -- the set of all IEEE-754 32-bit floating-point numbers
\item float64 -- the set of all IEEE-754 64-bit floating-point numbers
\end{itemize}

These are the two complex numbers:

\begin{itemize}
\item complex64 -- the set of all complex numbers with float32 real and imaginary parts
\item complex128 -- the set of all complex numbers with float64 real and imaginary parts
\end{itemize}

These are the two commonly used used aliases:

\begin{itemize}
\item byte -- alias for uint8
\item rune -- alias for int32
\end{itemize}

\subsection{String Type}

A string type is another most import primitive data type.  String type
represents string values.

\section{Constants}

A constant\index{constant} is an unchanging value.  Constants are
declared like variables, but with the \textit{const} keyword.
Constants can be character, string, boolean, or numeric values.
Constants cannot be declared using the \texttt{:=} syntax.

In Go, \textit{const} is a keyword introducing a name for a scalar
value such as 2 or 3.14159 or "scrumptious".  Such values, named or
otherwise, are called constants in Go.  Constants can also be created
by expressions built from constants, such as 2+3 or 2+3i or math.Pi/2
or ("go"+"pher").  Constants can be declared are at package level or
function level.

This is how to declare constants:

\begin{lstlisting}[numbers=none]
package main

import (
    "fmt"
)

const Freezing = true
const Pi = 3.14
const Name = "Tom"

func main() {
    fmt.Println(Pi, Freezing, Name)
}
\end{lstlisting}

You can also use the factored style declaration:

\begin{lstlisting}[numbers=none]
package main

import (
    "fmt"
)

const (
      Freezing = true
      Pi = 3.14
      Name = "Tom"
)

func main() {
    fmt.Println(Pi, Freezing, Name)
}

const (
      Freezing = true
      Pi = 3.14
      Name = "Tom"
)
\end{lstlisting}

Compiler throws an error if the constant is tried to assign a new
value:

\begin{lstlisting}[numbers=none]
package main

import (
    "fmt"
)

func main() {
    const Pi = 3.14
    Pi = 6.86
    fmt.Println(Pi)
}
\end{lstlisting}

The above program throws an error like this:

\begin{lstlisting}[numbers=none]
$ go run constants.go
constants:9:5: cannot assign to Pi
\end{lstlisting}

\subsection{iota}

The \textit{iota} keyword is used to define constants of incrementing numbers.
This simplify defining many constants. The values of iota is reset to \textit{0}
whenever the reserved word const appears. The value increments by one after each
line.

Consider this example:

\begin{lstlisting}[numbers=none]
// Token represents a lexical token.
type Token int

const (
    // Illegal represents an illegal/invalid character
    Illegal Token = iota

    // Whitespace represents a white space
    // (" ", \t, \r, \n) character
    Whitespace

    // EOF represents end of file
    EOF

    // MarkerID represents '\id' or '\id1' marker
    MarkerID

    // MarkerIde represents '\ide' marker
    MarkerIde
)
\end{lstlisting}

In the above example, the \texttt{Token} is custom type defined using
the primitive \textit{int} type.  The constants are defined using the
factored syntax (many constants within parenthesis).  There are
comments for each constant values.  Each constant value is be
incremented starting from \texttt{0}.  In the above
example, \texttt{Illegal} is \texttt{0}, Whitespace
is \texttt{1}, \texttt{EOF} is \texttt{2} and so on.

The \texttt{iota} can be used with expressions.  The expression will
be repeated.  Here is a good example taken from \textit{Effective
Go} (\url{https://go.dev/doc/effective_go#constants}):

\begin{lstlisting}[numbers=none]
type ByteSize float64

const (
    // ignore first value (0) by assigning to blank identifier
    _           = iota
    KB ByteSize = 1 << (10 * iota)
    MB
    GB
    TB
    PB
    EB
    ZB
    YB
)
\end{lstlisting}

Using \texttt{\_} (blank identifier) you can ignore a value,
but \textit{iota} increments the value.  This can be used to skip
certain values.  As you can see in the above example, you can use an
expression with \textit{iota}.

Iota is reset to \texttt{0} whenever the \textit{const} keyword appears in the
source code. This means that if you have multiple \textit{const} declarations in
a single file, \textit{iota} will start at \texttt{0} for each declaration. Iota
can only be used in \textit{const} declarations. It cannot be used in other
types of declarations, such as \textit{var} declarations. The value
of \textit{iota} is only available within the const declaration in which it is
used. It cannot be used outside of that declaration.

\subsubsection{Blank Identifier}

Sometimes you may need to ignore the value returned by a function.  Go
provides a special identifier called blank identifier\index{blank
identifier}\index{\_} to ignore any types of values. In Go,
underscore \texttt{\_} is the blank identifier.

Here is an example usage of blank identifier where the second value
returned by the function is discarded.

\begin{lstlisting}[numbers=none]
x, _ := someFunc()
\end{lstlisting}

Blank identifier can be used as import alias to invoke init function
without using the package.

\begin{lstlisting}[numbers=none]
import (
       "database/sql"

       _ "github.com/lib/pq"
)
\end{lstlisting}

In the above example, the \texttt{pq} package has some code which need
to be invoked to initialize the database driver provided by that
package.  And the exported functions within the above package is
supposed to be not used.

We have already seen another example where blank identifier if used
with \textit{iota} to ignore certain constants.

\section{Arrays}
\label{sec:arrays}

An array\index{array} is an ordered container type with a fixed number
of data.  In fact, the arrays are the foundation where slice is built.
We will study about slices in the next section.  Most of the time, you
can use slice instead of an array.

The number of values in the array is called the length of that array.
The array type \texttt{[n]T} is an array of \texttt{n} values of
type \texttt{T}.  Here are two example arrays:

\begin{lstlisting}[numbers=none]
colors := [3]string{"Red", "Green", "Blue"}
heights := [4]int{153, 146, 167, 170}
\end{lstlisting}

In the above example, the length of first array is \texttt{3} and the
array values are string data.  The second array contains \texttt{int}
values.  An array's length is part of its type, so arrays cannot be
re-sized.  So, if the length is different for two arrays, those are
distinct incompatible types. The built-in \texttt{len} function gives
the length of array.

Array values can be accessed using the index syntax, so the expression
\texttt{s[n]} accesses the nth element, starting from zero.

An array values can be read like this:

\begin{lstlisting}[numbers=none]
colors := [3]string{"Red", "Green", "Blue"}
i := colors[1]
fmt.Println(i)
\end{lstlisting}

Similarly array values can be set using index syntax.  Here is an
example:

\begin{lstlisting}[numbers=none]
colors := [3]string{"Red", "Green", "Blue"}
colors[1] = "Yellow"
\end{lstlisting}

Arrays need not be initialized explicitly. The zero value of an
array is a usable array with all elements zeroed.

\begin{lstlisting}[numbers=none]
var colors [3]string
colors[1] = "Yellow"
\end{lstlisting}

In this example, the values of colors will be empty strings (zero
value).  Later we can assign values using the index syntax.

There is a way to declare array literal without specifying the length.
When using this syntax variant, the compiler will count and set the
array length.

\begin{lstlisting}[numbers=none]
colors := [...]string{"Red", "Green", "Blue"}
\end{lstlisting}

In the chapter on control structures, we have seen how to use For loop
for iterating over slices.  In the same way, you can iterate over
array.

Consider this complete example:

\lstinputlisting{code/data-structures/colors.go}

If you save the above program in a file named \texttt{colors.go} and
run it, you will get output like this:

\begin{lstlisting}[numbers=none]
$ go run colors.go
Length: 3
0 Red
1 Green
2 Blue
\end{lstlisting}

In the above program, a string array is declared and initialized with
three string values.  In the 7th line, the length is printed and it
gives 3. The \texttt{range} clause gives index and value, where the
index starts from zero.

\section{Slices}

Slice\index{slice} is one of most important data structure in Go.
Slice is more flexible than an array.  It is possible to add and
remove values from a slice.  There will be a length for slice at any
time.  Though the length vary dynamically as the content value
increase or decrease.

The number of values in the slice is called the length of that slice.
The slice type \texttt{[]T} is a slice of type \texttt{T}.  Here are
two example slices:

\begin{lstlisting}[numbers=none]
colors := []string{"Red", "Green", "Blue"}
heights := []int{153, 146, 167, 170}
\end{lstlisting}

The first one is a slice of strings and the second slice is a slice of
integers.  The syntax is similar to array except the length of slice
is not explicitly specified.  You can use built-in \texttt{len}
function to see the length of slice.

Slice values can be accessed using the index syntax, so the expression
\texttt{s[n]} accesses the nth element, starting from zero.

A slice values can be read like this:

\begin{lstlisting}[numbers=none]
colors := []string{"Red", "Green", "Blue"}
i := colors[1]
fmt.Println(i)
\end{lstlisting}

Similary slice values can be set using index syntax.  Here is an
example:

\begin{lstlisting}[numbers=none]
colors := []string{"Red", "Green", "Blue"}
colors[1] = "Yellow"
\end{lstlisting}

Slices should be initialized with a length more than zero to access or
set values.  In the above examples, we used slice literal syntax for
that.  If you define a slice using \texttt{var} statement without
providing default values, the slice will be have a special zero value
called \texttt{nil}.

Consider this complete example:

\lstinputlisting{code/data-structures/nilslice.go}

In the above example, the value of slice \texttt{v} is \texttt{nil}.
Since the slice is nil, values cannot be accessed or set using the
index.  These operations are going to raise runtime error (index out
of range).

Sometimes it may not be possible to initialize a slice with some value
using the literal slice syntax given above.  Go provides a built-in
function named \texttt{make} to initialize a slice with a given length
and zero values for all items.  For example, if you want a slice with
3 items, the syntax is like this:

\begin{lstlisting}[numbers=none]
colors := make([]string, 3)
\end{lstlisting}

In the above example, a slice will be initialized with 3 empty strings
as the items.  Now it is possible to set and get values using the
index as given below:

\begin{lstlisting}[numbers=none]
colors[0] = "Red"
colors[1] = "Green"
colors[2] = "Blue"
i := colors[1]
fmt.Println(i)
\end{lstlisting}

If you try to set value at 3rd index (\texttt{colors[3]}), it's going
to raise runtime error with a message like this: "index out of range".
Go has a built-in function named \texttt{append} to add additional
values.  The append function will increase the length of the slice.

Consider this example:

\lstinputlisting{code/data-structures/appendslice.go}

In the above example, the slice length is increased by one after
append.  It is possible to add more values using \texttt{append}.  See
this example:

\lstinputlisting{code/data-structures/appendslicemore.go}

The above example append two values.  Though you can provide any
number of values to append.

You can use the "..." operator to expand a slice.  This can be used to
append one slice to another slice.  See this example:

\lstinputlisting{code/data-structures/appendsliceanother.go}

In the above example, the first slice is appended by all items in another slice.

\subsection{Slice Append Optimization}

If you append\index{slide!append} too many values to a slice using a
for loop, there is one optimization related that you need to be aware.

Consider this example:

\lstinputlisting{code/data-structures/appendfor.go}

If you run the above program, it's going to take few seconds to
execute.  To explain this, some understanding of internal structure of
slice is required.  Slice is implemented as a struct and an array
within.  The elements in the slice will be stored in the underlying
array.  As you know, the length of array is part of the array type.
So, when appending an item to a slice the a new array will be created.
To optimize, the \texttt{append} function actually created an array
with double length.

In the above example, the underlying array must be changed many times.
This is the reason why it's taking few seconds to execute.  The length
of underlying array is called the capacity of the slice.  Go provides
a way to initialize the underlying array with a particular length.
The \texttt{make} function has a fourth argument to specify the
capacity.

In the above example, you can specify the capacity like this:

\begin{lstlisting}[numbers=none]
v := make([]string, 0, 9000000)
\end{lstlisting}

If you make this change and run the program again, you can see that it
run much faster than the earlier code.  The reason for faster code is
that the slice capacity had already set with maximum required length.

\section{Maps}

Map\index{map} is another important data structure in Go.  We have
briefly discussed about maps in the Quick Start chapter.  As you know,
map is an implementation of hash table.  The hash table is available
in many very high level languages.  The data in map is organized like
key value pairs.

A variable of map can be declared like this:

\begin{lstlisting}[numbers=none]
var fruits map[string]int
\end{lstlisting}

To make use that variable, it needs to be initialized using \textit{make}
function.

\begin{lstlisting}[numbers=none]
fruits = make(map[string]int)
\end{lstlisting}

You can also initialize using the \textit{:=} syntax:

\begin{lstlisting}[numbers=none]
fruits := map[string]int{}
\end{lstlisting}

or with \textit{var} keywod:

\begin{lstlisting}[numbers=none]
var fruits = map[string]int{}
\end{lstlisting}

You can initialize map with values like this:

\begin{lstlisting}[numbers=none]
var fruits = map[string]int{
      "Apple":  45,
      "Mango":  24,
      "Orange": 34,
  }
\end{lstlisting}

After initializing, you can add new key value pairs like this:

\begin{lstlisting}[numbers=none]
fruits["Grape"] = 15
\end{lstlisting}

If you try to add values to maps without initializing, you will get an error
like this:

\begin{lstlisting}[numbers=none]
panic: assignment to entry in nil map
\end{lstlisting}

Here is an example that's going to produce panic error:

\begin{lstlisting}[numbers=none]
package main

func main() {
	var m map[string]int
	m["k"] = 7
}
\end{lstlisting}

To access a value corresponding to a key, you can use this syntax:

\begin{lstlisting}[numbers=none]
mangoCount := fruits["Mango"]
\end{lstlisting}

Here is an example:

\begin{lstlisting}[numbers=none]
package main

import "fmt"

func main() {
  var fruits = map[string]int{
      "Apple":  45,
      "Mango":  24,
      "Orange": 34,
  }
  fmt.Println(fruits["Apple"])
}
\end{lstlisting}

If the key doesn't exist, a zero value will be returned. For example, in the
below example, value of \texttt{pineappleCount} is going be \texttt{0}.

\begin{lstlisting}[numbers=none]
package main

import "fmt"

func main() {
  var fruits = map[string]int{
      "Apple":  45,
      "Mango":  24,
      "Orange": 34,
  }
  pineappleCount := fruits["Pineapple"]
  fmt.Println(pineappleCount)
}
\end{lstlisting}

If you need to check if the key exist, the above syntax can be modified to
return two values. The first one would be the actual value or zero value and the
second one would be a boolean indicating if the key exists.

\begin{lstlisting}[numbers=none]
package main

import "fmt"

func main() {
  var fruits = map[string]int{
      "Apple":  45,
      "Mango":  24,
      "Orange": 34,
  }
  _, ok := fruits["Pineapple"]
  fmt.Println(ok)
}
\end{lstlisting}

In the above program, the first returned value is ignored using a blank
identifier. And the value of \texttt{ok} variable is \texttt{false}. Normally,
you can use if condition to check if the key exists like this:

\begin{lstlisting}[numbers=none]
package main

import "fmt"

func main() {
	var fruits = map[string]int{
		"Apple":  45,
		"Mango":  24,
		"Orange": 34,
	}
	if _, ok := fruits["Pineapple"]; ok {
		fmt.Println("Key exists.")
	} else {
		fmt.Println("Key doesn't exist.")
	}
}
\end{lstlisting}

To see the number of key/value pairs, you can use the built-in \textit{len}
function.

\begin{lstlisting}[numbers=none]
package main

import "fmt"

func main() {
	var fruits = map[string]int{
		"Apple":  45,
		"Mango":  24,
		"Orange": 34,
	}
	fmt.Println(len(fruits))
}
\end{lstlisting}

The above program should print \texttt{3} as the number of items in the map.

To remove an item from the map, use the built-in \textit{delete} function.

\begin{lstlisting}[numbers=none]
package main

import "fmt"

func main() {
	var fruits = map[string]int{
		"Apple":  45,
		"Mango":  24,
		"Orange": 34,
	}
	delete(fruits, "Orange")
	fmt.Println(len(fruits))
}
\end{lstlisting}

The above program should print \texttt{2} as the number of items in the map
after deleting one item.

\section{Custom Data Types}

Apart from the built-in data types, you can create your own custom
data types\index{type!custom}.  The type keyword\index{type keyword}
can be used to create custom types.  Here is an example.

\begin{lstlisting}[numbers=none]
package main

import "fmt"

type age int

func main() {
     a := age(2)
     fmt.Println(a)
     fmt.Printf("Type: %T\n", a)
}
\end{lstlisting}

If you run the above program, the output will be like this:

\begin{lstlisting}[numbers=none]
$ go run age.go
2
Type: age
\end{lstlisting}

\subsection{Structs}

Struct\index{struct} is a composite type with multiple fields of
different types within the struct.  For example, if you want to
represent a person with name and age, the \texttt{struct} type will be
helpful.  The \texttt{Person} struct definition will look like this:

\begin{lstlisting}[numbers=none]
type Person struct {
    Name string
    Age  int
}
\end{lstlisting}

As you can see above, the \texttt{Person} struct is defined
using \texttt{type} and \texttt{struct} keywords.  Within the curly
brace, attributes with other types are defined.  If you avoid
attributes, it will become an empty struct.

Here is an example empty struct:

\begin{lstlisting}[numbers=none]
type Empty struct {
}
\end{lstlisting}

Alternatively, the curly brace can be in the same line.

\begin{lstlisting}[numbers=none]
type Empty struct {}
\end{lstlisting}

A struct can be initialized various ways.  Using a \texttt{var}
statement:

\begin{lstlisting}[numbers=none]
var p1 Person
p1.Name = "Tom"
p1.Age = 10
\end{lstlisting}

You can give a literal form with all attribute values:

\begin{lstlisting}[numbers=none]
p2 := Person{"Polly", 50}
\end{lstlisting}

You can also use named attributes.  In the case of named attributes,
if you miss any values, the default zero value will be initialized.

\begin{lstlisting}[numbers=none]
p3 := Person{Name: "Huck"}
p4 := Person{Age: 10}
\end{lstlisting}

In the next, we are going to learn about funtions and methods.  That
chapter expands the discussion about custom types bevior changes
through funtions associated with custom types called strcuts.

It is possible to embed structs inside other
structs\index{struct!embedding}.  Here is an example:

\begin{lstlisting}[numbers=none]
type Person struct {
     Name string
}

type Member struct {
     Person
     ID int
}
\end{lstlisting}

\section{Pointers}

When you are passing a variable as an argument to a function, Go
creates a copy of the value and send it.  In some situations, creating
a copy will be expensive when the size of object is large.  Another
scenario where pass by value is not feasible is when you need to
modify the original object inside the function.  In the case of pass
by value, you can modify it as you are getting a new object every
time.  Go supports another way to pass a reference to the original
value using the memory location or address of the
object\index{pointer}.

To get address of a variable, you can use \texttt{\&} as a prefix for
the variable.  Here in an example usage:

\begin{lstlisting}[numbers=none]
a := 7
fmt.Printf("%v\n", &a)
\end{lstlisting}

To get the value back from the address, you can use \texttt{*} as a
prefix for the variable.  Here in an example usage:

\begin{lstlisting}[numbers=none]
a := 7
b := &a
fmt.Printf("%v\n", *b)
\end{lstlisting}

Here is a complete example:

\lstinputlisting{code/data-structures/pointervalue.go}

A typical output will be like this:

\begin{lstlisting}[numbers=none]
0xc42000a340
0xc42000a348
0xc42000a340
\end{lstlisting}

As you can see above, the second output is different from the first
and third.  This is because a value is passed instead of a pointer.
And so when we are printing the address, it's printing the address of
the new variable.

In the functions chapter, the section about methods
(section~\ref{sec:methods}) explains the pointer receiver.

\subsection{new}

The built-in function \textit{new} can be used to allocate memory.  It
allocates zero values and returns the address of the given data type.

Here is an example:

\begin{lstlisting}[numbers=none]
name := new(string)
\end{lstlisting}

In this example, a \textit{string} pointer value is allocated with
zero value, in this case empty string, and assigned to a variable.

This above example is same as this:

\begin{lstlisting}[numbers=none]
var name *string
name = new(string)
\end{lstlisting}

In this one \textit{string} pointer variable is declared, but it's not
allocated with zeror value.  It will have \textit{nil} value and so it
cannot be dereferenced.  If you try to reference, without allocating
using the \textit{new} function, you will get an error like this:

\begin{lstlisting}[numbers=none]
panic: runtime error: invalid memory address or nil pointer
dereference
\end{lstlisting}

Here is another example using a custom type defined using a primitive
type:

\begin{lstlisting}[numbers=none]
type Temperature float64
name := new(Temperature)
\end{lstlisting}

\section{Exercises}

\textbf{Exercise 1:} Create a custom type for circle using float64
and define \texttt{Area} and \texttt{Perimeter}.

\textbf{Solution:}

\lstinputlisting[numbers=none]{code/data-structures/exercise1/circle.go}

The custom \texttt{Circle} type is created using the
built-in \texttt{float64} type.  It would be better if the circle is
defined using a struct.  Using struct helps to change the structure
later with additional attributes.  The struct will look like this:

\begin{lstlisting}[numbers=none]
type Circle struct {
    Radius float64
}
\end{lstlisting}

\textbf{Exercise 2:} Create a slice of structs where the struct represents
a person with name and age.

\textbf{Solution:}

\lstinputlisting[numbers=none]{code/data-structures/exercise2/persons.go}

\subsection{Additional Exercises}

Answers to these additional exercises are given in the Appendix A.

\textbf{Problem 1:} Write a program to record temperatures for different locations and check if it's freezing for a given place.

\textbf{Problem 2:} Create a map of world nations and details. The key could
be the country name and value could be an object with details including capital,
currency, and population.

\section*{Summary}

This chapter introduced various data structures in Go. These data structures
help organize data in a way that makes it easier to use and understand. The
chapter started with a section about zero values, which are values that are
assigned to variables when they are declared but not initialized. Then,
constants were explained in detail, including the \textit{iota} keyword, which
can be used to define incrementing constants. Next, the chapter briefly
explained arrays, which are data structures that can store a fixed number of
elements of the same type. Then, slices were explained, which are more flexible
data structures that can store a variable number of elements of the same type.
Slices are built on top of arrays, and they inherit many of the same properties.
Next, the chapter looked at how to define custom data types using existing
primitive types. Custom data types can be used to group together related data
and make it easier to work with. The struct was introduced as a way to define
custom data types. Structs can be used to store a collection of related data
items, such as the name, age, and address of a person. Pointers were also
covered. Pointers are variables that store the address of another variable.
Pointers can be used to access the value of another variable indirectly. The
next chapter will explain more about functions and methods.
