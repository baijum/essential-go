\cleardoublepage
\phantomsection
\chapter{Introduction}

\begin{quote}
\textit{I try to say everything at least three times: first, to
introduce it; second, to show it in context; and third, to show it in a
different context, or to review it.} --- Robert J. Chassell (An Introduction to
Programming in Emacs Lisp)
\end{quote}

Having proficiency in computer programming empowers you to tackle a wide range
of real-world challenges. Programming encompasses a process that begins with
formulating a computing problem and culminates in the creation of computer
programs or software. It is an integral part of a broader software development
process.

Programming entails analyzing, designing, and implementing software solutions.
Coding is the active process of implementing the software. At times, coding
involves utilizing multiple programming languages and incorporating other
technologies. Acquiring proficiency in a programming language is an essential
aspect of the skill set required for computer programming and software
development as a whole.

By utilizing a programming language, we construct instructions for computing
machines such as desktop computers, laptops, and mobile phones.

Today, there exists a multitude of programming languages with diverse features.
Choosing the appropriate programming language for a given problem is crucial.
Certain languages are better suited for specific challenges. This book serves as
an introduction to the Go programming language, offering valuable insights to
aid in making informed language selection decisions for your projects. If you
have already chosen to work with Go and seek to enhance your understanding, this
book provides an excellent introduction to Go programming.

Go, an open-source language developed by Google with significant contributions
from the community, has emerged as a popular choice. The project was initiated
in 2007 by Robert Griesemer, Rob Pike, and Ken Thompson and was subsequently
released as open-source software by Google in November 2009. Go has gained
widespread adoption across diverse problem domains, being embraced by numerous
organizations.

The design of the Go programming language draws inspiration from various
programming languages, such as C, Python, and occam. Go programs can run on many
operating systems including GNU/Linux, Windows and Mac OS X.

Before diving into learning Go programming, there are certain preparations that
you need to make. The following section will provide an explanation of these
necessary preparations.

This book is tailored for beginners who are eager to learn programming. While no
prior programming experience is required, readers are expected to have a basic
understanding of computers. The book comprehensively covers all the major topics
in the Go programming language, with each chapter focusing on one or two key
areas. Additionally, minor topics are introduced throughout the book at
appropriate intervals.

\section{Preparations}

Learning a programming language, such as Go, is akin to learning a natural
language like English, Chinese, or Spanish. It involves more than just
understanding the alphabet and grammar specific to that language. To become a
proficient programmer, there are several other important concepts and skills
that you need to acquire. While this book primarily focuses on Go programming,
we will also explore other essential topics that are crucial for your overall
programming journey.

In this section, we will delve into the installation of the Go compiler and the
setup of a suitable development environment. These steps are necessary to ensure
that you have the necessary tools in place to begin your Go programming
experience.

Text editors such as \href{https://vim-bootstrap.com}{Vim} and VS Code are
popular choices for writing Go programs. The files created using these text
editors are referred to as source files. To transform these source files into
executable programs, the Go compiler comes into play. Once the executable
program is generated, you can run it and obtain the desired output. Therefore,
it is essential to have a text editor and the Go compiler properly installed on
your computer system.

Utilizing a source code management system, such as Git, is highly advantageous.
It is strongly recommended to keep all your code under version control. There
are several public code hosting services available, such as GitHub, Bitbucket,
and GitLab, where you can store your code examples securely. By leveraging these
platforms, you can easily manage your code, collaborate with others, and track
changes effectively.

To install the Go compiler, please refer to the instructions specific to your
operating system. If you encounter any difficulties during the installation
process, don't hesitate to seek assistance from your friends or fellow
programmers. Once the Go compiler is installed, you can proceed to write a
simple Go program and run it to validate that the installation was successful.

\subsection{Linux Installation}

The Go project offers binaries for various major operating systems, including
GNU/Linux. If you are using a 64-bit GNU/Linux system, you can find the
appropriate binaries for your platform by visiting the following link:
\url{https://go.dev/dl}.

To download and install\index{install!linux} the Go compiler on a 64-bit
GNU/Linux system, please follow the instructions below. Before proceeding, make
sure that the Go compiler is not already installed by running the \texttt{go}
command. If the command returns \texttt{command not found...}, you can proceed
with the installation process.

To execute these commands, you need \textit{root} access or you can use
the \texttt{sudo} command. If you are unfamiliar with the process, you can seek
assistance from someone who has the necessary knowledge.

\begin{lstlisting}[numbers=none]
cd /tmp
wget https://go.dev/dl/go1.x.linux-amd64.tar.gz
tar -C /usr/local -zxvf go1.x.linux-amd64.tar.gz
\end{lstlisting}

The first line ensures that the current working directory is set to
the \texttt{/tmp} directory.

In the second line, replace the version number with the appropriate version
available on the \href{https://go.dev/dl}{Go downloads} website. This command
will download the 64-bit binary for GNU/Linux using the \textit{wget} command,
which is a command line download manager. Alternatively, you can
use \textit{curl} or any other download manager of your choice to download the
tar ball.

The third line extracts the downloaded tar ball into the \texttt{/usr/local/go}
directory.

You can now exit the \textit{root} user or stop using \textit{sudo}, depending
on the method you chose.

By default, Go packages are installed under the \texttt{\$HOME/go} directory.
However, you can override this directory by setting the \texttt{GOPATH}
environment variable. Any binaries installed using the \texttt{go install}
command will be placed in the \texttt{\$GOPATH/bin} directory.

To include the new binary locations in the \texttt{PATH} environment variable,
you can open the \texttt{\$HOME/.bashrc} file in a text editor and add the
following lines at the bottom:

\begin{lstlisting}[numbers=none]
export PATH=$HOME/go/bin:/usr/local/go/bin:$PATH
\end{lstlisting}

\subsection{Windows Installation}

For Windows users, there are separate installers\index{install!windows} (MSI
files) available for both 32-bit and 64-bit versions of the Go programming
language. The 32-bit version MSI file follows a naming convention similar to:
\textit{go1.x.y.windows-386.msi}, where \textit{x.y} represents the current version number.
Similarly, the 64-bit version MSI file is named
as \textit{go1.x.y.windows-amd64.msi}. It is important to replace \textit{x.y}
with the actual version number when downloading the appropriate installer file.

You can download the installers (MSI files) for the Go programming language from
the following website: \url{https://go.dev/dl}

This website provides the official distribution of Go, where you can find the
installers for various operating systems, including Windows. Simply navigate to
the provided link and select the appropriate MSI file for your Windows version
(32-bit or 64-bit) to begin the download.

Once you have downloaded the Go installer file (MSI), you can proceed with the
installation by following these steps:

\begin{enumerate}
\item Locate the downloaded MSI file on your computer.
\item Double-click on the MSI file to open it.
\item The installer will prompt you with a setup wizard that guides you through the installation process.
\item Follow the instructions provided by the setup wizard.
\item During the installation, you will be asked to choose the destination directory for the Go compiler.
\item By default, the installer will suggest the \texttt{C:\textbackslash Go} directory as the installation location.
\item You can either accept the default directory or choose a different location on your system.
\item Once you have selected the installation directory, click on the \textit{Install} or \textit{Next} button to proceed.
\item The installer will then copy the necessary files to the chosen directory.
\item After the installation is complete, you can close the installer.
\end{enumerate}

At this point, the Go compiler should be successfully installed on your system,
with the relevant files located in the specified installation directory
(e.g., \texttt{C:\textbackslash Go} by default).

Additionally, the installer automatically adds the \texttt{C:\textbackslash
Go\textbackslash bin} directory to the system PATH environment variable.
However, in order for this change to take effect, you may need to restart any
open command prompts or terminals on your system. This ensures that the Go
executables can be accessed from any command prompt or terminal window without
specifying the full path to the \texttt{bin} directory.

To download third-party packages from websites like GitHub.com, it is
recommended to create a directory where these packages will be stored. You can
create a directory named \texttt{mygo} at the \texttt{C:\textbackslash}
directory by following these steps:

\begin{enumerate}
\item Open a command prompt or terminal.
\item Type the following command and press Enter:
\end{enumerate}

\begin{lstlisting}[numbers=none]
mkdir C:\mygo
\end{lstlisting}

This will create a new directory named \texttt{mygo} directly under
the \texttt{C:\textbackslash} directory. You can use this directory to store and
manage the third-party packages you download for your Go projects.

To set the \textit{GOPATH} environment variable to point to the location where
you created the \texttt{mygo} directory (in this case,
\texttt{C:\textbackslash mygo}), follow these steps:

\begin{enumerate}
    \item Right-click on the \textit{This PC} or \textit{My Computer} icon on your desktop and select \textit{Properties}.
    \item In the System window, click on \textit{Advanced system settings} on the left-hand side.
    \item In the System Properties window, click on the \textit{Environment Variables} button.
    \item In the Environment Variables window, under the \textit{User variables} section, click on the \textit{New} button.
    \item Enter \textit{GOPATH} as the variable name.
    \item Enter the path to the \texttt{mygo} directory (in this case, \texttt{C:\textbackslash mygo}) as the variable value.
    \item Click \textit{OK} to save the changes.
\end{enumerate}

Once you have set the GOPATH environment variable, it will point to the
specified location, allowing Go to find and use the packages stored in the
\texttt{mygo} directory.

\subsection{Verifying Installation}

To verify that you have successfully installed Go, run the following command:

\begin{lstlisting}[numbers=none]
go version
\end{lstlisting}

This command will display the installed Go version if the installation was
successful. The output should looks like somthing like this:

\begin{lstlisting}[numbers=none]
go version go1.20.4 linux/amd64
\end{lstlisting}

Please note that the version number and platform may vary depending on the Go
version and the operating system you are using.

\subsection{Hello World!}

It is a common tradition in programming education to introduce a \textit{Hello
World} program as the first program. This program typically prints the message
\texttt{Hello World} to the console when executed.


Here is a simple \textit{Hello World} program. You can type the following source
code into your favorite text editor and save it as \texttt{hello.go}:

\begin{lstlisting}
package main

import "fmt"

func main() {
	fmt.Println("Hello, World!")
}
\end{lstlisting}

Make sure to save the file with the extension \textit{.go} to indicate that it
is a Go source code file.

Once you have saved the above source code into a file, you can open your command
line program (such as Terminal or Command Prompt). Then, navigate to the
directory where you saved the program code using the \texttt{cd} command. For
example:

\begin{lstlisting}[numbers=none]
cd /path/to/directory
\end{lstlisting}

Replace \texttt{/path/to/directory} with the actual path to the directory where
you saved the \texttt{hello.go} file.

Once you are in the correct directory, you can run the program by typing the
following command:

\begin{lstlisting}[numbers=none]
go run hello.go
\end{lstlisting}

Press Enter to execute the command. The program will be compiled and executed,
and you should see the output \texttt{Hello, World!} displayed in the command
line.

If you see the output as \texttt{Hello, World!}, congratulations! You have
successfully installed the Go compiler. In fact, the \texttt{go run} command
compiled your code into an executable format and then executed the program. The
next chapter will provide more detailed explanations about this example and
delve further into the concepts of Go programming.

Note: The Go Playground website, available at \url{https://go.dev/play}, serves
as a platform for publicly sharing Go source code. It also offers the
convenience of running programs directly within the browser.

\subsection{Using Git}

It is important to be comfortable using a source code management system, and
Git\index{git} is highly recommended. Creating an account on GitHub and
publishing your example code there would be beneficial. If you are new to Git
and have no prior experience, dedicating 2 to 3 days to learn Git would be
worthwhile.

\subsection{Using Command Line}

It is essential to be comfortable using command line\index{command line}
interfaces such as GNU Bash or PowerShell. There are numerous online tutorials
available on the Internet to learn shell commands. If you are unfamiliar with
command line usage, it is recommended to allocate a few days (around 3 to 4
days) to learn and familiarize yourself with the command line environment.

\section{Organization of Chapters}

The book is structured into the following chapters, which can be read in the
suggested order. The first six chapters are designed to be read sequentially,
while the remaining chapters can be read in any order you prefer.

\begin{description}
\item[Chapter 2: Quick Start] \hfill \\
This chapter serves as a tutorial introduction to the Go programming language.
It covers essential topics that form the foundation of Go programming, including
data types, variables, comments, for loops, range clauses, if statements,
functions, operators, slices, and maps. By studying these topics, readers will
gain a solid understanding of the fundamentals of Go programming.
\item[Chapter 3: Control Structures] \hfill \\
This chapter covers the different control structures available in the Go
programming language, including goto statements, if conditions, for loops, and
switch cases. Each of these topics is explained in detail, providing a
comprehensive understanding of how to effectively use these control structures
in Go programming.
\item[Chapter 4: Data Structures] \hfill \\
This chapter explores data structures in the Go programming language. It begins
by discussing arrays and then delves into slices, which are a more versatile
data structure built on top of arrays. The chapter also explains how to define
custom data types using existing primitive types and introduces the use of
structs for creating more complex custom data types. Pointers are also covered,
along with structs, slices, and maps in the language.
\item[Chapter 5: Functions] \hfill \\
This chapter provides a comprehensive explanation of functions in Go. It covers
various aspects such as sending input parameters and returning values. The
chapter also explores variadic functions and anonymous functions. Additionally,
there is a brief introduction to methods in Go.
\item[Chapter 6: Objects] \hfill \\
This chapter delves into the concept of objects and interfaces in Go and their
practical applications. Interfaces hold significant importance in Go as they
contribute to robust design. The chapter covers the concept of empty interfaces
and provides an overview of pointer receivers and their significance.
Additionally, the chapter explores type assertions and type switches in Go.
\item[Chapter 7: Concurrency] \hfill \\
In this chapter, the concurrency features of Go are explained in detail.
Depending on the problem at hand, you can choose between channels and other
synchronization techniques. The chapter covers the usage of goroutines and
channels, highlighting their importance in concurrent programming. It also
explores topics such as Waitgroups, Select statements, buffered channels, and
channel direction. Additionally, the chapter provides an introduction to the
usage of the sync.Once function.
\item[Chapter 8: Packages] \hfill \\
In this chapter, the concept of Go packages is thoroughly explained. Packages
serve as fundamental building blocks for creating reusable Go programs. The
chapter covers various aspects such as creating packages, documenting packages,
and the process of publishing packages. It also delves into the topic of modules
and their usage in managing dependencies. Additionally, the chapter provides
insights on moving types across packages during the refactoring process.
\item[Chapter 9: Input/Output] \hfill \\
In this chapter, the various input/output functionalities in Go are explored.
The chapter covers topics such as handling command line arguments and
interactive input. It introduces the usage of the flag package for handling
command line options and arguments. Additionally, the chapter provides insights
into various string formatting techniques used in Go.
\item[Chapter 10: Testing] \hfill \\
In this chapter, the process of writing tests using the testing package in Go is
explained. The chapter covers important concepts such as marking tests as
failures, logging, skipping tests, and running tests in parallel. Additionally,
the chapter briefly introduces the concept of sub-tests, providing a glimpse
into its usage.
\item[Chapter 11: Tooling] \hfill \\
In this chapter, the Go tool is introduced and all its commands are explained in
detail. Practical examples are provided for each command to illustrate their
usage. The chapter covers important commands such as building and running
programs, running tests, formatting code, and displaying documentation.
Additionally, the chapter briefly mentions a few other useful tools that can
enhance your development workflow.
\end{description}

In addition to the solved exercises, each chapter includes additional problems
for further practice. The answers to these additional problems can be found in
Appendix A.

Lastly, the book concludes with an index at the end, which serves as a helpful
reference for locating specific topics and concepts throughout the book.

\section{Suggestions to Use this Book}

It is important to follow the instructions provided in this chapter to set up
your system with the Go compiler and the necessary environment. If you encounter
any difficulties during the setup process, don't hesitate to seek assistance
from your friends or colleagues. Additionally, utilizing a source code
management system like Git will greatly facilitate the management of your code.
You can use it to write exercises, solve additional problems, and keep your code
under version control for easy tracking and collaboration.

I would recommend against simply copying and pasting code from the book.
Instead, I encourage you to actively type out each example provided. By manually
typing the code, you will gain a better understanding of the syntax and
structure of the language, which will help you become more familiar with it more
quickly. Additionally, typing out the code will improve your muscle memory and
reinforce your learning process. So, take the time to engage with the code
actively and type it out yourself for a more effective learning experience.

It is recommended to read the first six chapters of the book in order, starting
from the Introduction to Interfaces. These initial chapters lay the foundation
and cover important concepts that are built upon in the later chapters. Reading
them sequentially will provide a logical progression of knowledge and
understanding.

However, once you have completed the first six chapters, the remaining chapters
can be read in any order based on your specific interests or needs. Each chapter
in the later part of the book focuses on a specific topic or aspect of Go
programming, and they are designed to be relatively independent of each other.
Feel free to explore the chapters that align with your interests or are relevant
to your current programming goals.

By following this approach, you will establish a solid understanding of the
fundamentals through the initial chapters and then have the flexibility to delve
into specific topics of your choice in the later chapters.

\section*{Summary}

In this chapter, we provided an introduction to the Go programming language. We
covered essential topics that are crucial for becoming a proficient programmer.
Our discussion touched upon various aspects that are important to understand in
order to develop strong programming skills.

We emphasized that learning Go programming involves more than just understanding
the language syntax. To become a good programmer, it is necessary to grasp a
wide range of concepts and skills. While our focus in this book is on Go
programming, we acknowledged that there are several other topics and areas of
knowledge that are valuable for programmers to explore.

Furthermore, we highlighted the significance of installing the Go compiler and
setting up a suitable development environment to facilitate the learning
process. We also encouraged the use of source code management systems like Git
to manage and organize your code effectively.

By introducing these essential elements in this chapter, we aim to provide a
solid foundation for your journey in learning Go programming.

Following the introduction to Go programming, we proceeded to discuss the
organization of chapters in this book. We outlined the structure and sequence of
topics covered, emphasizing that the first six chapters should be read in order.
These initial chapters serve as a foundation for the subsequent content, forming
the building blocks for a comprehensive understanding of Go programming.

Additionally, we provided some helpful suggestions on how to effectively utilize
this book for learning Go. These recommendations are intended to enhance your
learning experience and maximize your understanding of the language and its
concepts.

Moving forward, the next chapter will provide a quick start to programming with
the Go language. It will offer practical examples and hands-on exercises to help
you get started with writing Go programs. This chapter aims to provide a smooth
transition from theoretical concepts to practical application, enabling you to
gain firsthand experience with Go programming.
