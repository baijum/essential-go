\cleardoublepage
\phantomsection
\chapter{Introduction}

\begin{quote}
\textit{I try to say everything at least three times: first, to
introduce it; second, to show it in context; and third, to show it in a
different context, or to review it.} --- Robert J. Chassell (An Introduction to
Programming in Emacs Lisp)
\end{quote}

Computer programming skill helps you to solve many real-world problems.
Programming is a process starting from the formulation of a computing problem to
producing computer programs (software). Computer programming is part of a more
extensive software development process.

Programming involves analysis, design, and implementation of the software.
Coding is the activity of implementing software. Sometimes coding involves more
than one programming language and use of other technologies. Learning a
programming language is a crucial part of the skill required for computer
programming or software development in general.

Using a programming language, we are preparing instructions for a computing
machine. The computing machine includes a desktop computer, laptop, and mobile
phone.

There are many programming languages in use today with different feature sets.
You should learn to pick the right programming language for the problem at hand.
Some languages are more suitable for specific problems. This book provides an
introduction to the Go programming language. Studying the Go programming should
help you to make the right decision about programming language choice for your
projects. If you have already decided to use Go, this book should give an
excellent introduction to Go programming.

Go, also commonly referred to as Golang, is a general-purpose programming
language. Go was initially developed at Google in 2007 by Robert Griesemer, Rob
Pike, and Ken Thompson. Go was publicly released as an open source software in
November 2009 by Google.

Many other programming languages including C, Python, and occam has inspired the
design of the Go programming language. Go programs can run on many operating
systems including GNU/Linux, Windows and Mac OS X.

You require some preparations to learn Go programming. The next section explains
the preparations required.

This book is for beginners who want to learn to programme. The readers are
expected to have a basic knowledge of computers. This book covers all the major
topics in the Go programming language. Each chapter in this book covers one or
two major topics. However many minor topics are introduced intermittently.

This chapter provides an introduction to the language, preparations required to
start practicing the exercises in this book, organizing code and walk through of
remaining chapters. Few suggestions for learning Go using this book is given
towards the end of this chapter.

\section{Preparations}

Learning a natural language like English, Chinese, and Spanish is not just
understanding the alphabet and grammar of that particular language. Similarly,
there are many things that you need to learn to become a good programmer. Even
though this book is going to focus on Go programming; now we are going to see
some other topics that you need to learn. This section also explains the
installation of the Go compiler and setting up the necessary development
environment.

Text editors like Notepad, Notepad++, Gedit, and Vim can be used to write Go
programs. The file that you create using the text file is called source file.
The source file text is UTF-8 encoded. The Go compiler creates executable
programs from the source file. You can run the executable program and get the
output. So, you need a text editor and Go compiler installed in your system.

Depending on your operating system, follow the instruction given below to
install the Go compiler. If you have difficulty following this, you may get some
help from your friends to do this step. Later we write a simple Go program and
run it to validate the steps.

You can use any text editor to write code. If you are not familiar with any text
editor, consider using Vim. You can bootstrap Vim configuration for Go
programming language from this webste: \url{https://vim-bootstrap.com}.

Using a source code management system like Git would be helpful. Keeping all
your code under version control is highly recommended. You can use a public code
hosting service like GitHub, Bitbucket, and GitLab to store your examples.

\subsection{Linux Installation}

Go project provides binaries for major operating systems including GNU/Linux.
You can find 64 bit binaries for GNU/Linux here: \url{https://go.dev/dl}

The following commands will download and install\index{install!linux} Go
compiler in a 64 bit GNU/Linux system. Before performing these steps, ensure Go
compiler is not installed by running \texttt{go} command. If it
prints \textit{command not found...}, you can proceed with these steps.

These commands must be run as \textit{root} or through \textit{sudo}. If you do
not know how to do it, you can get help from somebody else.

\begin{lstlisting}[numbers=none]
cd /tmp
wget https://go.dev/dl/go1.x.linux-amd64.tar.gz
tar -C /usr/local -zxvf go1.x.linux-amd64.tar.gz
\end{lstlisting}

The first line ensure that current working directory is the \texttt{/tmp}
directory.

You should change the version number in the second line and it's going to
download the 64 bit binary for GNU/Linux. The \texttt{wget} is a command line
download manager. Alternatively you can use \texttt{curl} or any other download
manager to download the tar ball.

The third line extract the downloaded tar ball into \texttt{/usr/local/go}
directory.

Now you can exit from the \textit{root} user or stop using \textit{sudo}.

By default Go packages are installed under \texttt{\$HOME/go} directory. This
directory can be overridden using \texttt{GOPATH} environment variable. Any
binaries installed using \texttt{go install} and \texttt{go get} commands goes
into \texttt{\$GOPATH/bin} directory.

You can also update PATH environment variable to include new binary locations.
Open the \texttt{\$HOME/.bashrc} file in a text editor and enter this lines at
the bottom.

\begin{lstlisting}[numbers=none]
export PATH=$HOME/go/bin:/usr/local/go/bin:$PATH
\end{lstlisting}

\subsection{Windows Installation}

There are separate installers\index{install!windows} (MSI files) available for
32 bit and 64 bit versions of Windows. The 32 bit version MSI file will be named
like this: \textit{go1.x.y.windows-386.msi} (Replace \texttt{x.y} with the
current version). Similarly for 64 bit version, the MSI file will be named like
this: \textit{go1.x.y.windows-amd64.msi} (Replace \texttt{x.y} with the current
version).

You can download the installers (MSI files) from here:
\url{https://go.dev/dl}

After downloading the installer file, you can open the MSI file by double
clicking on that file. This should prompts few things about the installation of
the Go compiler. The installer place the Go related files in
the \texttt{C:\textbackslash{}Go} directory.

The installer also put the \texttt{C:\textbackslash{}Go\textbackslash{}bin}
directory in the system \texttt{PATH} environment variable. You may need to
restart any open command prompts for the change to take effect.

You also need to create a directory to download third party packages from
github.com or similar sites. The directory can be created at
\texttt{C:\textbackslash{}mygo} like this:

\begin{lstlisting}[numbers=none]
C:\> mkdir C:\mygo
\end{lstlisting}

After this you can set \texttt{GOPATH} environment variable to point to this
location.

\begin{lstlisting}[numbers=none]
C:\> go env -w GOPATH=C:\mygo
\end{lstlisting}

You can also append \texttt{C:\textbackslash{}mygo\textbackslash{}bin} into
the \texttt{PATH} environment variable.

If you do not know how to set environment variable, just do a Google search
for: \textit{set windows environment variable}.

\subsection{Hello World!}

It's kind of a tradition in teaching programming to introduce a \textit{Hello
World} program as the first program. This program normally prints
a \textit{Hello World} to the console when running.

Here is our hello world program. You can type the source code given below to
your favorite text editor and save it as \texttt{hello.go}.

\begin{lstlisting}[caption=Hello World! (hello.go)]
package main

import "fmt"

func main() {
	fmt.Println("Hello, World!")
}
\end{lstlisting}

Once you saved the above source code into a file. You can open your command line
program (bash or cmd.exe) then change to the directory where you saved the
program code and run the above program like this:

\begin{lstlisting}[numbers=none]
$ go run hello.go
Hello, World!
\end{lstlisting}

If you see the output as \texttt{Hello, World!}, congratulations! Now you have
successfully installed Go compiler. In fact, the \texttt{go run} command
compiled your code to an executable format and then run that program. The next
chapter explains more about this example.

\subsection{Using Git}

You should be comfortable using a source code management system. As mentioned
above Git\index{git} would be a good choice. You can create an account in GitHub
and publish your example code there. If you do not have any prior experience,
you can spend 2 to 3 days to learn Git.

\subsection{Using Command Line}

You should be comfortable using command line\index{command line} interfaces like
GNU Bash or PowerShell. There are many online tutorials available in the
Internet to learn shell commands. If you do not have any prior experience, you
can spend few days (3 to 4 days) to learn command line usage.

%% \section{Organizing Code}

%% As mentioned above, you can place project source files under the
%% workspace directory\index{organizing code}.  The workspace directory
%% is by default pointing to \texttt{\$HOME/go}.  This can be changed
%% using \texttt{\$GOPATH} environment variable.

%% The workspace directory struture looks like this:

%% \begin{lstlisting}[numbers=none]
%% mygo
%% |-- bin
%% |-- pkg
%% `-- src
%% \end{lstlisting}

%% Under the workspace directory there will be three sub-directories
%% named \texttt{bin}, \texttt{pkg} and \texttt{src}.  The \texttt{bin}
%% directory contains executable binaries created from Go programs.  We
%% have already added this \texttt{\$GOPATH/bin} directory
%% to \texttt{PATH} environment variable.  This will help us to execute
%% programs directly without specifying the full path.  The \texttt{src}
%% directory contains the source files.  The \texttt{pkg} directory
%% contains package objects used by go tool to create the final
%% executable binaries.

%% Under the \texttt{src} directory, you can place your code anywhere.
%% However there are certain conventions followed to organize the files.

%% If you are using GitHub for hosting code, you can create a directory
%% structure under workspace like this:

%% \begin{lstlisting}[numbers=none]
%% src/github.com/<username>/<projectname>
%% \end{lstlisting}

%% Replace the \texttt{<username>} with your GitHub username or
%% organization name and \texttt{<projectname>} with the name of the
%% project.

%% For example, if the username is \texttt{baijum} and project
%% is \texttt{introduction}, the layout will be like this:

%% \begin{lstlisting}[numbers=none]
%% mygo
%% |-- bin
%% |-- pkg
%% `-- src
%%     `-- github.com
%%        `-- baijum
%%            `-- introduction
%% \end{lstlisting}

%% The example hello world program we introduced earlier has been pushed
%% into GitHub here: \url{https://github.com/baijum/introduction}.  You
%% can get this code into the workspace using \texttt{go get} command.
%% The \texttt{go get} command download source repositories and places
%% them in the workspace.  Since the \texttt{\$PATH} has been updated to
%% include \texttt{\$HOME/go/bin} and \texttt{go get} place a binary
%% under that, you can execute the program using \texttt{introduction}
%% command:

%% \begin{lstlisting}[numbers=none]
%% $ go get github.com/baijum/introduction
%% $ introduction
%% Hello, World!
%% \end{lstlisting}

%% Alternatively, you can run this program from the source
%% location (\texttt{\$HOME/go/src/github.com/baijum/introduction}).

%% \begin{lstlisting}[numbers=none]
%% $ cd $HOME/go/src/github.com/baijum/introduction
%% $ go run hello.go
%% Hello, World!
%% \end{lstlisting}

%% We will walk through the hello world example in the next chapter.

%% The \texttt{go} command line program is called Go tool\index{tool}
%% (\url{https://pkg.go.dev/cmd/go}).  The Go tool understands the layout
%% of a workspace (\texttt{\$GOPATH} directory).  The Go tool has many
%% sub-commands.  To see all commands use \texttt{go help} and to see
%% help of a particular command use \texttt{go help <command>}.  There is
%% a chapter dedicated about Go tooling later in this book.

\section{Organization of Chapters}

The rest of the book is organized into the following chapters. It is recommended
to read the first six chapters in order -- that's until the chapter on
interfaces. The remaining chapters can be read in any order.

\begin{description}
\item[Chapter 2: Quickstart] \hfill \\
This chapter provides a tutorial introduction to the language. It introduce few
basic topics in Go programming language. The topics include Data Types,
Variables, Comments, For Loop, Range Clause, If, Function, Operators, Slices,
and Maps.
\item[Chapter 3: Control Structures] \hfill \\
This chapter cover the various control structures like \textit{goto}, \textit{if
condition}, \textit{for loop} and \textit{switch case} in the language. It goes
into details of each of these topics.
\item[Chapter 4: Data Structures] \hfill \\
This chapter cover data structures. The chapter starts with arrays. Then slices,
the more useful data structure built on top of array is explained. Then we
looked at how to define custom data types using existing primitive types. The
struct is introduced which is more useful to create custom data types. Pointer
is also covered. like \textit{struct}, \textit{slice} and \textit{map} in the
language.
\item[Chapter 5: Functions] \hfill \\
This chapter explained all the major aspects of functions in Go. The chapter
covered how to send input parameters and return values. It also explained about
variadic function and anonymous function. This chapter briefly also covered
methods.
\item[Chapter 6: Interfaces] \hfill \\
This chapter explained the concept of interfaces and it's uses. Interface is an
important concept in Go. Understanding interfaces and properly using it makes
the design robust. The chapter covered empty interface. Also, briefly explained
about pointer receiver and its significance. Type assertions and type switches
are also explained.
\item[Chapter 7: Concurrency] \hfill \\
This chapter explained concurrency features of Go. Based on your problem, you
can choose channels or other synchronization techniques. This chapter covered
goroutines and channels usage. It covered Waitgroups, Select statement. It also
covered buffered channels, channel direction. The chapter also touched
upon \textit{sync.Once} function usage.
\item[Chapter 8: Packages] \hfill \\
This chapter explained the Go package in detail. Package is one of building
block of a reusable Go program. This chapter explained about creating packages,
documenting packages, and finally about publish packages. The chapter also
covered modules and its usage. Finally it explained moving types across packages
during refactoring.
\item[Chapter 9: Input/Output] \hfill \\
This chapter discussed about various input/output related functionalities in Go.
The chapter explained using command line arguments and interactive input. The
chaptered using \textit{flag} package. It also explained about various string
formatting techniques.
\item[Chapter 10: Testing] \hfill \\
This chapter explained writing tests using the \textit{testing} package. It
covered how to mark a test as a failure, logging, skipping, and parallel
running. Also, it briefly touched upon sub-tests.
\item[Chapter 11: Tooling] \hfill \\
This chapter introduced the Go tool. All the Go commands were explained in
detail. Practical example usage was also given for each command. The chapter
coverd how to build and run programs, running tests, formatting code, and
displaying documentation. It also touched upon few other handy tools.
\end{description}

In addition to the solved exercises, each chapter contains additional problems.
Answers to these additional problems are given in Appendix A.

And finally there is an index at the end of the book.

\section{Suggestions to Use this Book}

Make sure to setup your system with Go compiler and the environment as explained
in this chapter. If you are finding it very difficult, you may get help from
your friends to setup the environment. Use source code management system like
Git to manage your code. You can write exercises and solve additional problems
and keep it under version control.

I would suggest not to copy \& paste code from the E-book. Rather you can type
every examples in this book. This will help you to familiarize the syntax much
quickly.

The first 6 chapters, that is from Introduction to Interfaces should be read in
order. The remaining chapters are based on the first 6 chapters. So, the chapter
7 onward can be read in any order.

\section*{Summary}

This chapter provided an introduction to Go programming language. We briefly
discussed about topics required to become a good programmer.
%% Later we covered source code organization in the workspace.

Then we covered chapter organization in this book. And finally few suggestions
for readers are given. The next chapter provides a quickstart to programming
with Go language.
