\cleardoublepage
\phantomsection
\chapter{Input/Output}

\begin{quote}
\textit{for out of the abundance of the heart the mouth speaketh.} -- Bible
\end{quote}

Users interact with software systems through various input/output
\index{input/output} mechanisms.  Some of the commonly used mechanisms are these:

\begin{itemize}
  \item web browser for web applications using various controllers/widgets
  \item mobile for mobile applications
  \item shell for command line applications
  \item desktop application with native controllers/widgets
\end{itemize}

To provide these kind of user interactions, you will be required to use
specialized libraries. If the standard library doesn't provide what you are
looking for, you may need to use third party libraries. This chapter cover basic
mechanisms provided at the language level which you can use for input/output.

We have already seen some of the basic input/output techniques in the last few
chapters. This chapter will go though more input/output mechanisms available in
Go.

\section{Command Line Arguments}

The command line\index{command line} arguments as user input and
console for output is the way command line programs are designed.
Sometimes output will be other files and devices.  You can access all
the command line arguments using the \texttt{Args} array/slice
attribute available in \texttt{os} package.  Here is an example:

\lstinputlisting[numbers=none]{code/io/commandline.go}

You can run this program with minimum two additional arguments:

\begin{lstlisting}[numbers=none]
$ go build cmdline.go
$ ./cmdline one -two
[cmdline one -two]
./cmdline
one
-two
\end{lstlisting}

As you can see, it's difficult to parse command line arguments like this. Go
standard library has a package named \textit{flag} which helps to easily parse
command line arguments. This chapter has a section to explain the \textit{flag}
package.

\section{Files}

Reading and writing data to files is a common I/O operation in computer
programming. You can manipulate files using the \textit{os} and \textit{io}
packages. It works with both text files and binary files. For the simplicity of
this section, all the examples given here works with text files.

Consider there exists a file named \texttt{poem.txt} with this text:

\lstinputlisting[numbers=none]{code/io/readpoem/poem.txt}

Here is a program to read the the whole file and print:

\lstinputlisting[caption=Read whole file]{code/io/readpoem/readpoem.go}

When you run this program, you will get the whole text as output. In the line
number 10, the \texttt{Open} function from the \texttt{os} package is called to open
the file for reading. It returns a file
descriptor\footnote{\url{https://en.wikipedia.org/wiki/File_descriptor}} and
error. In the line number 14, an infinite loop is stared to read the content.
Line 15 initialize a slice of bytes of length 50. The \texttt{fd.Read} method
reads the given length of characters and writes to the given slice. It returns
the number of characters read and error. The \texttt{io.EOF} error is returned
when end of file is reached. This is used as the condition to break the loop.

Here is a program to write some text to a file:

\lstinputlisting[caption=Write to file]{code/io/writefile/writefile.go}

In th line number 9, the \textit{Create} function from the \textit{os} package
is called open the file for writing. It returns a file descriptor and error. In
the line number 14, the \textit{Write} method is give a slice of bytes to write.
After running the program you can see the text in the \texttt{hello.txt} file.

\begin{lstlisting}[numbers=none]
$ go run writefile.go
$ cat hello.txt
Hello, World!
\end{lstlisting}

\section{Standard Streams}

Standard streams\footnote{\url{https://en.wikipedia.org/wiki/Standard_streams}}
are input and output communication channels between a computer program and its
environment. The three input/output connections are called standard input
(stdin), standard output (stdout) and standard error (stderr).

Stdin, Stdout, and Stderr are open files pointing to the standard input,
standard output, and standard error file descriptors.

The \textit{fmt} package has functions to read values
interactively\index{input!scanf}.

Here is an example:

\lstinputlisting[caption=Read name and print]{code/io/readprint.go}

The \textit{Scanf} function reads the standard input. The first argument is the
format and the second one is the pointer variable. The value read from standard
input cab be accessed using the given variable.

You can run the above program in different ways:

\begin{lstlisting}[numbers=none]
$ go run code/io/readprint.go
Enter your name: Baiju
Your name: Baiju
$ echo "Baiju" |go run code/io/readprint.go
Enter your name: Your name: Baiju
$ go run code/io/readprint.go << EOF
> Baiju
> EOF
Enter your name: Your name: Baiju
$ echo "Baiju" > /tmp/baiju.txt
$ go run code/io/readprint.go < /tmp/baiju.txt
Enter your name: Your name: Baiju
\end{lstlisting}

As you can see from this program, the \textit{Printf} function writes to
standard output and the \textit{Scanf} reads the standard input. Go can also
writes to standard error output stream.

The \textit{io} package provides a set of interfaces and functions that allow
developers to work with different types of input and output streams.

Consider a use case to convert everything that comes to standard input to
convert to upper case. This can be achieved by reading all standard input
using \texttt{io.ReadAll} and converting to upper case.  Here is code:

\lstinputlisting[caption=Convert standard input to upper case]{code/io/uppercase/uppercase.go}

You can run this program similar to how you did with the previous program.

You can use \textit{fmt.Fprintf} with \textit{os.Stderr} as the first argument
to write to standard error.

\begin{lstlisting}[numbers=none]
fmt.Fprintf(os.Stderr, "This goes to standard error: %s", "OK")
\end{lstlisting}

Alternatively, you can call \textit{WriteString} method of \textit{os.Stderr}:

\begin{lstlisting}[numbers=none]
os.Stderr.WriteString("This goes to standard error")
\end{lstlisting}

\section{Using flag Package}

As you have noticed before \texttt{os.Args} attribute in the \textit{os} package
provides access to all command line arguments.
The \textit{flag}\index{input!flag} package provides an easy way to parse
command line arguments.

You can define string, boolean, and integer flags among others using
the \textit{flag} package..

Here is an integer flag declaration:

\begin{lstlisting}[numbers=none]
var pageCount = flag.Int("count", 240, "number of pages")
\end{lstlisting}

The above code snippet defines an integer flag with name
as \texttt{count} and it is stored in a variable with the name
as \texttt{pageCount}.  The type of the variable is \textit{*int}.
Similar to this integer flag, you can defines flags of other types.

Once all the flags are defined, you can parse it like this:

\begin{lstlisting}[numbers=none]
flag.Parse()
\end{lstlisting}

The above \texttt{Parse} function call parse the command line
arguments and store the values in the given variables.

Once the flags are parsed, you can dereference it like this:

\begin{lstlisting}[numbers=none]
fmt.Println("pageCount: ", *pageCount)
\end{lstlisting}

To access non-flag arguments:

\begin{lstlisting}[numbers=none]
flag.Args()
\end{lstlisting}

The above call returns a the arguments as a slice of strings.  It
contains arguments not parsed as flags.

Cobra is a third party package providing a simple interface to create
command line interfaces.  Cobra also helps to generate applications
and command files.  Many of the most widely used Go projects are built
using Cobra.  This is the Cobra
website: \url{https://github.com/spf13/cobra}

\section{String Formatting}

Go supports many string format\index{string!format} options.  To
get the default format of any value, you can use \texttt{\%v} as the
format string.  Here is an example which print formatted values
using \texttt{\%v}:

\lstinputlisting[caption=Default format]{code/io/formatting/default.go}

The \texttt{\%T} shows the type of the value.  The output of the above
program will be like this.

\begin{lstlisting}[numbers=none]
Value: Baiju, Type: string
Value: 7, Type: int
Value: 7, Type: uint
Value: 7, Type: int8
Value: true, Type: bool
Value: 7, Type: float64
Value: (1+6i), Type: complex128
\end{lstlisting}

If you use a \texttt{\%+v} as the format string for struct it shows
the field names.  See this example:

\lstinputlisting[caption=Default format for struct]{code/io/formatting/defaultstruct.go}

If you run the above program, the output is going to be like this:

\begin{lstlisting}[numbers=none]
Value: {76.45 blue}
Value with fields: {radius:76.45 color:blue}
Type: main.Circle
\end{lstlisting}

As you can see from the output, in the first line \texttt{\%v} doesn't
show the fields.  But in the second line, \texttt{\%+v} shows the
struct fields.

The \texttt{\%\#v} shows the representation of the value. Here is a
modified version of above program to print few values of primitive
type.

\lstinputlisting[caption=Representation format]{code/io/formatting/defaultrepr.go}

\begin{lstlisting}[numbers=none]
Value: "Baiju", Type: string
Value: 7, Type: int
Value: 0x7, Type: uint
Value: 7, Type: int8
Value: true, Type: bool
Value: 7, Type: float64
Value: (1+6i), Type: complex128
\end{lstlisting}

As you can see in the representation, strings are written within
quotes.  You can also see representation of few other primitive types.

If you want a literal \texttt{\%} sign, use two \texttt{\%} signs next
to each other.  Here is a code snippet:

\begin{lstlisting}[numbers=none]
        fmt.Println("Tom scored 92%% marks")
\end{lstlisting}

The default string representation of custom types can be changed by
implementing \texttt{fmt.Stringer} interafce.  The interface
definition is like this:

\begin{lstlisting}[numbers=none]
type Stringer interface {
        String() string
}
\end{lstlisting}

As per the \texttt{Stringer} interface, you need to create
a \texttt{String} function which return a string.  Now the value
printed will be whatever returned by that function.  Here is an
example:

\lstinputlisting[caption=Custom representation using Stringer]{code/io/custom/custom.go}

The output of the above program will be like this:

\begin{lstlisting}[numbers=none]
30.46 degree Celsius
30.46 degree Celsius
30.46 degree Celsius
main.Temperature{Value:30.456, Unit:"Celsius"}
\end{lstlisting}

\section{Exercises}

{\bfseries Exercise 1:} Write a program to read length and width of a
rectangle through command line arguments and print the area.
Use \texttt{-length} switch to get length and \texttt{-width} switch
to get width.  Represent the rectangle using a struct.

\textbf{Solution:}

\lstinputlisting[numbers=none]{code/io/rectangle/rectangle.go}

You can run the program like this:

\begin{lstlisting}[numbers=none]
$ go run rectangle.go -length 2.5 -width 3.4
Area:  8.5
\end{lstlisting}

\subsection{Additional Exercises}

Answers to these additional exercises are given in the Appendix A.

\textbf{Problem 1:} Write a program to format a complex number as used in mathematics.  Example: \texttt{2 + 5i}

Use a struct like this to define the complex number:

\begin{lstlisting}[numbers=none]
type Complex struct {
    Real float64
    Imaginary float64
}
\end{lstlisting}


\section*{Summary}

This chapter discussed the input/output (I/O) features of the Go programming
language. It explained how to use command line arguments and interactive input,
and how to use the \textit{flag} package to parse command line arguments. It
also explained various string formatting techniques.
