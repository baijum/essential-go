\cleardoublepage
\phantomsection
\chapter{Objects}

\begin{quote}
\textit{Program to an interface, not an implementation.} -- Design Patterns by Gang of Four
\end{quote}


In the chapter on functions, we have also learned about methods.  A
method is a function associated with a type, more specifically a
concrete type.  As you know, an object is an instance of a type.  In
general, methods define the behavior of an object.

Interfaces\index{interface} in Go provide a formal way to specify the
behavior of an object.  In layman's terms, Interface is like a
blueprint which describes an object.  So, Interface is considered as
an abstract type, commonly referred to as interface type.

Small interfaces with one or two methods are common in Go.

Here is a \texttt{Geometry} interface which defines two methods:

\begin{lstlisting}[numbers=none]
type Geometry interface {
    Area() float64
    Perimeter() float64
}
\end{lstlisting}

If any type satisfy this interface - that is define these two methods
which returns float64 - then, we can say that type is implementing
this interface.  One difference with many other languages with
interface support and Go is that, in Go implementing an interface
happens implicitly.  So, no need to explicitly declare a type is
implementing a particular interface.

To understand this idea, consider this \texttt{Rectangle} struct type:

\begin{lstlisting}[numbers=none]
type Rectangle struct {
    Width float64
    Height float64
}

func (r Rectangle) Area() float64 {
    return r.Width * r.Height
}

func (r Rectangle) Perimeter() float64 {
    return 2*(r.width + r.height)
}
\end{lstlisting}

As you can see above, the above Rectangle type has two methods named
Area and Perimeter which returns float64.  So, we can say Rectangle is
implementing the \texttt{Geometry} interface.  To elaborate the
example further, we will create one more implementation:

\begin{lstlisting}[numbers=none]
type Circle struct {
    Radius float64
}

func (c Circle) Area() float64 {
    return 3.14 * c.Radius * c.Radius
}

func (c Circle) Perimeter() float64 {
    return 2 * 3.14 * c.Radius
}
\end{lstlisting}

Now we have two separate implementations for the \texttt{Geometry}
interface.  So, if anywhere the \texttt{Geometry} interface type is
expected, you can use any of these implementations.

Let's define a function which accepts a \texttt{Geometry} type and
prints area and perimeter.

\begin{lstlisting}[numbers=none]
func Measure(g Geometry) {
    fmt.Println("Area:", g.Area())
    fmt.Println("Perimeter:", g.Perimeter())
}
\end{lstlisting}

When you call the above function, you can pass the argument as an
object of \texttt{Geometry} interface type.  Since both
\texttt{Rectangle} and \texttt{Circle} satisfy that interface, you can
use either one of them.

Here is a code which call \texttt{Measure} function with
\texttt{Rectangle} and \texttt{Circle} objects:

\begin{lstlisting}[numbers=none]
r := Rectangle{Width: 2.5, Height: 4.0}
c := Circle{Radius: 6.5}
Measure(r)
Measure(c)
\end{lstlisting}

\section{Type with Multiple Interfaces}

In Go, a type can implement more than one interface.  If a type has
methods that satisfy different interfaces, we can say that that type
is implementing those interfaces.

Consider this interface:

\begin{lstlisting}[numbers=none]
type Stringer interface {
     String() string
}
\end{lstlisting}

In previous section, there was a Rectangle type declared with two
methods.  In the same package, if you declare one more method like
below, it makes that type implementing Stringer interface in addition
to the Geometry interface.

\begin{lstlisting}[numbers=none]
func (r Rectangle) String() string {
    return fmt.Sprintf("Rectangle %vx%v", r.Width * r.Height)
}
\end{lstlisting}

Now the \texttt{Rectangle} type conforms to both \texttt{Geometry}
interface and \texttt{Stringer} interface.

\section{Empty Interface}

The empty interface\index{interface!empty} is the interface type that
has no methods.  Normally the empty interface will be used in the
literal form: \texttt{interface{}}.  All types satisfy empty interface.
A function which accepts empty interface, can receive any type as the
argument.  Here is an example:

\begin{lstlisting}[numbers=none]
func blackHole(v interface{}) {
}

blackHole(1)
blackHole("Hello")
blackHole(struct{})
\end{lstlisting}

In the above code, the \texttt{blackHole} functions accepts an empty
interface. So, when you are calling the function, any type of argument
can be passed.

The \texttt{Println} function in the \texttt{fmt} package is variadic
function which accepts empty interfaces.  This is how the function
signature looks like:

\begin{lstlisting}[numbers=none]
func Println(a ...interface{}) (n int, err error) {
\end{lstlisting}

Since the \texttt{Println} accepts empty interfaces, you could pass
any type arguments like this:

\begin{lstlisting}[numbers=none]
fmt.Println(1, "Hello", struct{})
\end{lstlisting}

\section{Pointer Receiver}

In the chapter on Functions, you have seen that the methods can use a
pointer receiver\index{method!pointer receiver}.  Also we understood
that the pointer receivers are required when the object attributes
need be to modified or when passing large size data.

Consider the implementation of \texttt{Stringer} interface here:

\begin{lstlisting}[numbers=none]
type Temperature struct {
     Value float64
     Location string
}

func (t *Temperature) String() string {
     o := fmt.Sprintf("Temp: %.2f Loc: %s", t.Value, t.Location)
     return o
}
\end{lstlisting}

In the above example, the \texttt{String} method is implemented using
a pointer receiver.  Now if you define a function which accepts the
\texttt{fmt.Stringer} interface, and want the \texttt{Temperature}
object, it should be a pointer to \texttt{Temperature}.

\begin{lstlisting}[numbers=none]
func cityTemperature(v fmt.Stringer) {
    fmt.Println(v.String())
}

func main() {
    v := Temperature{35.6, "Bangalore"}
    cityTemperature(&v)
}
\end{lstlisting}

As you can see, the \texttt{cityTemperature} function is called with a
pointer.  If you modify the above code and pass normal value, you will
get an error.  The below code will produce an error as pointer is not
passed.

\begin{lstlisting}[numbers=none]
func main() {
    v := Temperature{35.6, "Bangalore"}
    cityTemperature(v)
}
\end{lstlisting}

The error message will be something like this:

\begin{lstlisting}[numbers=none]
cannot use v (type Temperature) as type fmt.Stringer in argument to
cityTemperature: Temperature does not implement fmt.Stringer (String
method has pointer receiver)
\end{lstlisting}

\section{Type Assertions}

In some cases, you may want to access the underlying concrete value
from the interface value.  Let's say you define a function which
accepts an interface value and want access attribute of the concrete
value.  Consider this example:

\begin{lstlisting}[numbers=none]
type Geometry interface {
    Area() float64
    Perimeter() float64
}

type Rectangle struct {
    Width float64
    Height float64
}

func (r Rectangle) Area() float64 {
    return r.Width * r.Height
}

func (r Rectangle) Perimeter() float64 {
    return 2*(r.width + r.height)
}

func Measure(g Geometry) {
    fmt.Println("Area:", g.Area())
    fmt.Println("Perimeter:", g.Perimeter())
}
\end{lstlisting}

In the above example, if you want to print the width and and height
from the \texttt{Measure} function, you can use type assertions.

Type assertion\index{type!assertion} gives the underlying concrete
value of an interface type.  In the above example, you can access the
rectangle object like this:

\begin{lstlisting}[numbers=none]
    r := g.(Rectangle)
    fmt.Println("Width:", r.Width)
    fmt.Println("Height:", r.Height)
\end{lstlisting}

If the assertion fail, it will trigger a panic.

Type assertion has an alternate syntax where it will not panic if
assertion fail, but gives one more return value of boolean type.  The
second return value will be \texttt{true} if assertion succeeds
otherwise it will give \texttt{false}.

\begin{lstlisting}[numbers=none]
    r, ok := g.(Rectangle)
    if ok {
        fmt.Println("Width:", r.Width)
        fmt.Println("Height:", r.Height)
    }
\end{lstlisting}

If there are many types that need to be asserted like this, Go
provides a type switches which is explained in the next section.

\section{Type Switches}

As you have seen in the previous section, type assertions gives access
to the underlying value.  But if there any many assertions need to be
made, there will be lots \texttt{if} blocks.  To avoid this, Go
provides type switches\index{switch!type}.

\begin{lstlisting}[numbers=none]
    switch v := g.(type) {
    case Rectangle:
        fmt.Println("Width:", v.Width)
        fmt.Println("Height:", v.Height)
    case Circle:
        fmt.Println("Width:", v.Radius)
    case default:
        fmt.Println("Unknown:")
    }
\end{lstlisting}

In the above example, type assertion is used with switch cases.  Based
on the type of \texttt{g}, the case is executed.

Note that the \textit{fallthrough} statement does not work in type
switch.

\section{Exercises}

\textbf{Exercise 1:} Using the \texttt{Marshaller} interface, make the marshalled output of the
\texttt{Person} object given here all in upper case.

\begin{lstlisting}[numbers=none]
type Person struct {
    Name  string
    Place string
}
\end{lstlisting}

\textbf{Solution:}

\lstinputlisting[numbers=none]{code/interfaces/marshal.go}

\subsection{Additional Exercises}

Answers to these additional exercises are given in the Appendix A.

\textbf{Problem 1:} Implement the built-in \texttt{error} interface for a custom data type.
This is how the \texttt{error} interface is defined:

\begin{lstlisting}[numbers=none]
type error interface {
    Error() string
}
\end{lstlisting}


\section*{Summary}

This chapter explained the concept of interfaces and it's uses.
Interface is an important concept in Go.  Understanding interfaces and
properly using it makes the design robust.  The chapter covered empty
interface.  Also, briefly explained about pointer receiver and its
significance.  Type assertions and type switches are also explained.
