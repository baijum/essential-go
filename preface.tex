\cleardoublepage
\phantomsection
\chapter*{Preface}
\addcontentsline{toc}{chapter}{\numberline{}Preface}

\begin{quote}
\textit{For surely there is an end; and thine expectation shall not be cut
off.} --- Proverbs 23:18
\end{quote}

If you are looking for a simple and powerful general-purpose
programming language, \textit{Go} would be a great
choice.  \textit{Go} is an open source software from Google and the
project has lots of community contributions.  The project was started
in 2007 by Robert Griesemer, Rob Pike, and Ken Thompson.  However, it
was publicly released by Google as an open source software in November
2009.  \textit{Go} is used by many organizations in different problem
domains.  Remember, the search-friendly word for \textit{Go}
is \textit{Golang}.

This book is an introductory book that covers the basics of the Go
programming language.  If you are new to programming, I hope this book
is going to be useful for you.  If you studied programming a long time
ago and are looking for a refresher program, I would suggest you take
a look at the \url{https://scratch.mit.edu} website.
The \textit{Scratch} programming language website will be helpful for
those who have never programmed seriously.  Once you become
comfortable with the fundamental concepts of programming, you can come
back to this book.

I started with Python programming around 2003, and since then, I have
worked on Python for more than a decade.  One of my old colleagues
introduced me to \textit{Go} in 2013.  It was a refreshing experience
when I started with Go programming.  There are lots of differences
between \textit{Python} and \textit{Go}, but I was impressed with the
simplicity of \textit{Go} in the initial days.  The learning curve was
very smooth compared to some other languages that I tried.

When I became interested in Go programming, one of the first things
that I decided was to write a book about it.  Writing has been my
passion for a long time.  I started writing a blog in 2004 on
LiveJournal.  I wrote my first book about Zope component architecture
in 2007.  Writing is a fun activity; even though, sometimes it becomes
hectic.  Since this is a self-published book, there was no pressure on
me, so it took quite some time to get this book in good shape.

I have conducted many \textit{Go} workshops in different parts of
India.  During the workshops, I always wanted something more to offer
to the participants, and I hope this book will fill that vacuum.

Software development is not just about programming languages.  There
are other skills you need to acquire, such as learning how to use a
good text editor/IDE, a version control system, and the operating
system of your choice.  Moreover, pursue the learning of other
languages and technologies throughout your career.  The Go programming
language leans towards a more object oriented style programming.  You
could also learn functional programming languages like Scheme and
Haskell.

Beyond technical skills, your business domain knowledge and soft
skills are essential for your career growth.  Writing more about that
would be out of scope in this context, so let me leave it there and
wish you a successful career at this moment.
\thispagestyle{plain}

\vspace*{.2in}
Baiju Muthukadan\\
Kozhikode, Kerala, India\\
May 2023
