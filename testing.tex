\cleardoublepage
\phantomsection
\chapter{Testing}

\begin{quote}
\textit{Test as You Fly, Fly as You Test.} --- NASA
\end{quote}

Writing automated tests helps you to improve the quality and
reliability of software.  This chapter is about writing automated
tests in Go.  The standard library contains a package
named \textit{testing} to write tests\index{test}.  Also, the built-in
Go tool has a test runner to run tests.

Consider this test for a \texttt{Hello} function which takes a string
as input and returns another string.  The expected output string
should start with "Hello," and ends with the input value followed by
an exclamation.

\lstinputlisting[caption=Test for Hello]{code/testing/hello/hello_test.go}

In the first line, the package name is \texttt{hello} which is the
same as the package where the function is going to be defined.  Since
both test and actual code is in the same package, the test can access
any name within that package irrespective whether it is an exported
name or not.  At the same, when the actual problem is compiled using
the \texttt{go build} command, the test files are ignored.  The build
ignores all files with the name ending \texttt{\_test.go}.  Sometimes
these kinds of tests are called white-box test as it is accessing both
exported and unexported names within the package.  If you only want to
access exported names within the test, you can declare the name of the
test package with \texttt{\_test} suffix.  In this case, that would
be \texttt{hello\_test}, which should work in this case as we are only
testing the exported function directly.  However, to access those
exported names -- in this case, a function -- the package should
import explicitly.

The line no. 5 starts the test function declaration.  As per the test
framework, the function name should start with \texttt{Test} prefix.
The prefix is what helps the test runner to identify the tests that
should be running.

The input parameter \texttt{t} is a pointer of
type \texttt{testing.T}. The \texttt{testing.T} type provides
functions to make the test pass or fail.  It also gives functions to
log messages.

In the line no. 6 the \texttt{Hello} function is called with "Tom" as
the input string.  The return value is assigning to a variable
named \texttt{out}.

In line no. 7 the actual output value is checked against the expected
output value.  If the values are not matching, the \texttt{Fail}
function is getting called.  The particular function state is going to
be marked as a failure when the \texttt{Fail} function is getting
called.

The test is going to pass or fail based on the implementation of
the \texttt{Hello} function.  Here is an implementation of the
function to satisfy the test:

\lstinputlisting[caption=Hello function]{code/testing/hello/hello.go}

As you can see in the above function definition, it takes a string as
an input argument.  A new string is getting constructed as per the
requirement, and it returns that value.

Now you can run the test like this:

\begin{lstlisting}[numbers=none]
$ go test
PASS
ok      _/home/baiju/hello     0.001s
\end{lstlisting}

If you want to see the verbose output, use the \texttt{-v} option:

\begin{lstlisting}[numbers=none]
$ go test -v
=== RUN   TestHello
--- PASS: TestHello (0.00s)
PASS
ok      _/home/baiju/hello     0.001s
\end{lstlisting}

\section{Failing a Test}

To fail a test, you need to explicitly call \texttt{Fail} function
provided by the value of \texttt{testing.T} type.  As we have seen
before, every test function has access to a \texttt{testing.T} object.
Usually, the name of that value is going to write as \texttt{t}. To
fail a test, you can call the \texttt{Fail} function like this:

\begin{lstlisting}[numbers=none]
t.Fail()
\end{lstlisting}

The test is going be a failure when the \texttt{Fail} function is
getting called.  However, the remaining code in the same test function
continue to execute.  If you want to stop executing the further lines
immediately, you can call \texttt{FailNow} function.

\begin{lstlisting}[numbers=none]
t.FailNow()
\end{lstlisting}

Alternatively, there are other convenient functions which give similar
behavior along with logging message.  The next section discusses
logging messages.

\section{Logging Message}

The \texttt{testing.T} has two functions for logging, one with default
formatting and the other with the user-specified format.  Both
functions accept an arbitrary number of arguments.

The \texttt{Log} function formats its arguments using the default
formats available for any type.  The behavior is similar
to \texttt{fmt.Println} function.  So, you can change the formatted
value by implementing the \texttt{fmt.Stringer} interface:

\begin{lstlisting}[numbers=none]
type Stringer interface {
        String() string
}
\end{lstlisting}

You need to create a method named \texttt{String} which returns a
string for your custom types.

Here is an example calling \texttt{Log} with two arguments:

\begin{lstlisting}[numbers=none]
t.Log("Some message", someValue)
\end{lstlisting}

In the above function call, there are only two arguments given, but
you can pass any number of arguments.

The log message going to print if the test is failing.  The verbose
mode, the \texttt{-v} command-line option, also log the message
irrespective of whether a test fails or not.

The \texttt{Logf} function takes a string format followed by arguments
expected by the given string format.  Here is an example:

\begin{lstlisting}[numbers=none]
t.Logf("%d no. of lines: %s", 34, "More number of lines")
\end{lstlisting}

The \texttt{Logf} formats the values based on the given format.
The \texttt{Logf} is similar to \texttt{fmt.Printf} function.

\section{Failing with Log Message}

Usually, logging and marking a test as failure happens simultaneously.
The \texttt{testing.T} has two functions for logging with failing, one
with default formatting and the other with the user-specified format.
Both functions accept an arbitrary number of arguments.

The \texttt{Error} function is equivalent to calling \texttt{Log}
followed by \texttt{Fail}.  The function signature is similar
to \texttt{Log} function.

Here is an example calling \texttt{Error} with two arguments:

\begin{lstlisting}[numbers=none]
t.Error("Some message", someValue)
\end{lstlisting}

Similar to \texttt{Error} function, the \texttt{Errorf} function is
equivalent to calling \texttt{Logf} followed by \texttt{Fail}.  The
function signature is similar to \texttt{Logf} function.

The \texttt{Errorf} function takes a string format followed by
arguments expected by the given string format.  Here is an example:

\begin{lstlisting}[numbers=none]
t.Errorf("%d no. of lines: %s", 34, "More number of lines")
\end{lstlisting}

The \texttt{Errorf} formats the values based on the given format.

\section{Skipping Test}

When writing tests, there are situations where particular tests need
not run.  Some tests might have written for a specific environment.
The criteria for running tests could be CPU architecture, operating
system or any other parameter.  The \textit{testing} package has
functions to mark test for skipping.

The \texttt{SkipNow} function call marks the test as having been
skipped.  It stops the current test execution.  If the test has marked
as failed before skipping, that particular test is yet considered to
have failed.  The \texttt{SkipNow} function doesn't accept any
argument.  Here is a simple example:

\lstinputlisting[caption=Skipping test]{code/testing/skip/hello_skip_test.go}

If you run the above code on a Linux system, you can see the test has
skipped.  The output is going to be something like this:

\begin{lstlisting}[numbers=none]
$ go test . -v
=== RUN   TestHello
--- SKIP: TestHello (0.00s)
PASS
ok      _/home/baiju/skip      0.001s
\end{lstlisting}

As you can see from the output, the test has skipped execution.

There are two more convenient functions similar to \texttt{Error}
and \texttt{Errorf}.  Those functions are \texttt{Skip}
and \texttt{Skipf}.  These functions help you to log a message before
skipping.  The message could be the reason for skipping the test.

Here is an example:

\begin{lstlisting}[numbers=none]
t.Skip("Some reason message", someValue)
\end{lstlisting}

The \texttt{Skipf} function takes a string format followed by
arguments expected by the given string format.  Here is an example:

\begin{lstlisting}[numbers=none]
t.Skipf("%d no. of lines: %s", 34, "More number of lines")
\end{lstlisting}

The \texttt{Skipf} formats the values based on the given format.

\section{Parallel Running}

You can mark a test to run in parallel. To do so, you can call the
\texttt{t.Parallel} function.  The test is going to run in parallel to other
tests marked parallel.

\section{Sub Tests}

The Go testing package allows you to group related tests together in a
hierarchical form. You can define multiple related tests under a single-parent
test using the `Run` method.

To create a subtest, you use the \texttt{t.Run()} method. The \texttt{t.Run()} method takes
two arguments: the name of the subtest and the body of the subtest. The body of
the subtest is a regular Go function.

For example, the following code creates a subtest called \texttt{foo}:

\begin{lstlisting}[numbers=none]
func TestBar(t *testing.T) {
  t.Run("foo", func(t *testing.T) {
    // This is the body of the subtest.
  })
}
\end{lstlisting}

Subtests are reported separately from each other. This means that if a subtest
fails, the test runner will report the failure for that subtest only. The parent
test will still be considered to have passed.

Subtests can be used to test different aspects of a function or method. For
example, you could use subtests to test different input values, different output
values, or different error conditions.

Subtests can also be used to test different implementations of a function or
method. For example, you could use subtests to test a function that is
implemented in Go and a function that is implemented in C.

Subtests are a powerful feature of the Go testing package. They can be used to
organize your tests, make them easier to read and maintain, and test different
aspects of your code.

%% \section{TestMain}
%% \section{Test Runner}
%% \section{Test Flags}

\section{Exercises}

{\bfseries Exercise 1:} Create a package with a function to return the
sum of two integers and write a test for the same.

\textbf{Solution:}

\textit{sum.go}:

\lstinputlisting[numbers=none]{code/testing/sum/sum.go}

\textit{sum\_test.go}:

\lstinputlisting[numbers=none]{code/testing/sum/sum_test.go}

\subsection{Additional Exercises}

Answers to these additional exercises are given in the Appendix A.

\textbf{Problem 1:} Write a program to fail test and not continue with remaining tests.

\section*{Summary}

This chapter explained writing tests using the \textit{testing} package. It
covered how to mark a test as a failure, logging, skipping, and parallel
running. Also, it briefly touched upon sub-tests.
