\cleardoublepage
\phantomsection
\chapter{Tooling}

\begin{quote}
\textit{We become what we behold. We shape our tools, and thereafter our tools shape us.} --- Marshall McLuhan
\end{quote}

Good support for lots of useful tools is another strength of Go.
Apart from the built-in tools, there any many other community-built
tools.  This chapter covers the built-in Go tools and few other
external tools.

The built-in Go tools can access through the \texttt{go} command.
When you install the Go compiler (\texttt{gc}); the \texttt{go} tool
is available in the path.  The \texttt{go} tool has many commands.
You can use the \texttt{go} tool to compile Go programs, run test
cases, and format source files among other things.

\section{Getting Help}

The go tool is self-documented\index{tool!help}.  You can get help
about any commands easily.  To see the list of all commands, you can
run the \texttt{"help"} command.  For example, to see help
for \texttt{build} command, you can run like this:

\begin{lstlisting}[numbers=none]
go help build
\end{lstlisting}

The help command also provides help for specific topics like
"buildmode", "cache", "filetype", and "environment" among other
topics.  To see help for a specific topic, you can run the command
like this:

\begin{lstlisting}[numbers=none]
go help environment
\end{lstlisting}

\section{Basic Information}

\subsection{Version}

When reporting bugs, it is essential to specify the Go version\index{version}
number and environment details. The Go tool gives access to this information
through the following commands.

To get version information, run this command\index{tool!version}:

\begin{lstlisting}[numbers=none]
go version
\end{lstlisting}

The output should look something like this:

\begin{lstlisting}[numbers=none]
go version go1.20.4 linux/amd64
\end{lstlisting}

As you can see, it shows the version number followed by operating
system and CPU architecture.

\subsection{Environment}

To get environment variables, you can run this command\index{tool!env}:

\begin{lstlisting}[numbers=none]
go env
\end{lstlisting}

The output should display all the environment variables used by the Go
tool when running different commands.

A typical output will look like this:

\begin{lstlisting}[numbers=none]
GO111MODULE=""
GOARCH="amd64"
GOBIN=""
GOCACHE="/home/baiju/.cache/go-build"
GOENV="/home/baiju/.config/go/env"
GOEXE=""
GOEXPERIMENT=""
GOFLAGS=""
GOHOSTARCH="amd64"
GOHOSTOS="linux"
GOINSECURE=""
GOMODCACHE="/home/baiju/go/pkg/mod"
GONOPROXY=""
GONOSUMDB=""
GOOS="linux"
GOPATH="/home/baiju/go"
GOPRIVATE=""
GOPROXY="https://proxy.golang.org,direct"
GOROOT="/usr/local/go"
GOSUMDB="sum.golang.org"
GOTMPDIR=""
GOTOOLDIR="/usr/local/go/pkg/tool/linux_amd64"
GOVCS=""
GOVERSION="go1.20.4"
GCCGO="gccgo"
GOAMD64="v1"
AR="ar"
CC="gcc"
CXX="g++"
CGO_ENABLED="1"
GOMOD="/dev/null"
GOWORK=""
CGO_CFLAGS="-O2 -g"
CGO_CPPFLAGS=""
CGO_CXXFLAGS="-O2 -g"
CGO_FFLAGS="-O2 -g"
CGO_LDFLAGS="-O2 -g"
PKG_CONFIG="pkg-config"
GOGCCFLAGS="-fPIC -m64 -pthread -Wl,--no-gc-sections -fmessage-length=0
  -fdebug-prefix-map=/tmp/go-build1378738152=/tmp/go-build
  -gno-record-gcc-switches"
\end{lstlisting}

\subsection{List}

The \textit{list} command provides meta information about packages.
Running without any arguments shows the current packages import path.
The \textit{-f} helps to extract more information, and it can specify
a format.  The \textit{text/template} package syntax can be used to
specify the format.

The struct used to format has many attributes, here is a subset:

\begin{itemize}
\item \textit{Dir} -- directory containing package sources
\item \textit{ImportPath} -- import path of package in dir
\item \textit{ImportComment} -- path in import comment on package statement
\item \textit{Name} -- package name
\item \textit{Doc} -- package documentation string
\item \textit{Target} -- install path
\item \textit{GoFiles} -- list of \texttt{.go} source files
\end{itemize}

Here is an example usage:

\begin{lstlisting}[numbers=none]
$ go list -f '{{.GoFiles}}' text/template
[doc.go exec.go funcs.go helper.go option.go template.go]
\end{lstlisting}

\section{Building and Running}

To compile a program, you can use the \texttt{build}
command\index{tool!build}.  To compile a package, first change to the
directory where the program is located and run the \texttt{build}
command:

\begin{lstlisting}[numbers=none]
go build
\end{lstlisting}

You can also compile Go programs without changing the directory.  To
do that, you are required to specify the package location in the
command line.  For example, to
compile \texttt{github.com/baijum/introduction} package run the
command given below:

\begin{lstlisting}[numbers=none]
go build github.com/baijum/introduction
\end{lstlisting}

If you want to set the executable binary file name, use
the \texttt{-o} option:

\begin{lstlisting}[numbers=none]
go build -o myprog
\end{lstlisting}

If you want to build and run at once, you can use the \texttt{"run"}
command:

\begin{lstlisting}[numbers=none]
go run program.go
\end{lstlisting}

You can specify more that one Go source file in the command line
arguments:

\begin{lstlisting}[numbers=none]
go run file1.go file2.go file3.go
\end{lstlisting}

Of course, when you specify more than one file names, only one "main"
function should be defined among all of the files.

\subsection{Conditional Compilation}

Sometimes you need to write code specific to a particular operating
system.  In some other case the code for a particular CPU
architecture.  It could be code optimized for that particular
combination.  The Go build tool supports conditional compilation using
build constraints.  The Build constraint is also known as build tag.
There is another approach for conditional compilation using a naming
convention for files names.  This section is going to discuss both
these approaches.

The build tag should be given as comments at the top of the source
code.  The build tag comment should start like this:

\begin{lstlisting}[numbers=none]
// +build
\end{lstlisting}

The comment should be before package documentation and there should be
a line in between.

The space is \textit{OR} and comma is \textit{AND}.  The exclamation
character is stands for negation.

Here is an example:

\begin{lstlisting}[numbers=none]
// +build linux,386
\end{lstlisting}

In the above example, the file will compile on 32-bit x86 Linux
system.

\begin{lstlisting}[numbers=none]
// +build linux darwin
\end{lstlisting}

The above one compiles on Linux or Darwin (Mac OS).

\begin{lstlisting}[numbers=none]
// +build !linux
\end{lstlisting}

The above runs on anything that is not Linux.

The other approach uses file naming convention for conditional
compilation.  The files are ignore if it doesn't match the target OS
and CPU architecture, if any.

This compiles only on Linux:

\begin{lstlisting}[numbers=none]
stat_linux.go
\end{lstlisting}

This one on 64 bit ARM linux:

\begin{lstlisting}[numbers=none]
os_linux_arm64.go
\end{lstlisting}

\section{Running Test}

The Go tool has a built-in test runner\index{tool!test}.  To run tests
for the current package, run this command:

\begin{lstlisting}[numbers=none]
go test
\end{lstlisting}

To demonstrate the remaining commands, consider packages organized
like this:

\begin{lstlisting}[numbers=none]
.
|-- hello.go
|-- hello_test.go
|-- sub1
|   |-- sub1.go
|   `-- sub1_test.go
`-- sub2
    |-- sub2.go
    `-- sub2_test.go
\end{lstlisting}

If you run \texttt{go test} from the top-level directory, it's going
to run tests in that directory, and not any sub directories.  You can
specify directories as command line arguments to \texttt{go test}
command to run tests under multiple packages simultaneously.
In the above listed case, you can run all tests like this:

\begin{lstlisting}[numbers=none]
go test . ./sub1 ./sub2
\end{lstlisting}

Instead of listing each packages separates, you can use the ellipsis
syntax:

\begin{lstlisting}[numbers=none]
go test ./...
\end{lstlisting}

The above command run tests under current directory and its child
directories.

By default \texttt{go test} shows very few details about the tests.

\begin{lstlisting}[numbers=none]
$ go test ./...
ok      _/home/baiju/code/mypkg   0.001s
ok      _/home/baiju/code/mypkg/sub1      0.001s
--- FAIL: TestSub (0.00s)
FAIL
FAIL    _/home/baiju/code/mypkg/sub2      0.003s
\end{lstlisting}

In the above results, it shows the name of failed test.  But details
about other passing tests are not available.  If you want to see
verbose results, use the \texttt{-v} option.

\begin{lstlisting}[numbers=none]
$ go test ./... -v
=== RUN   TestHello
--- PASS: TestHello (0.00s)
PASS
ok      _/home/baiju/code/mypkg   0.001s
=== RUN   TestSub
--- PASS: TestSub (0.00s)
PASS
ok      _/home/baiju/code/mypkg/sub1      0.001s
=== RUN   TestSub
--- FAIL: TestSub (0.00s)
FAIL
FAIL    _/home/baiju/code/mypkg/sub2      0.002s
\end{lstlisting}

If you need to filter tests based on the name, you can use
the \texttt{-run} option.

\begin{lstlisting}[numbers=none]
$ go test ./... -v -run Sub
testing: warning: no tests to run
PASS
ok      _/home/baiju/code/mypkg   0.001s [no tests to run]
=== RUN   TestSub
--- PASS: TestSub (0.00s)
PASS
ok      _/home/baiju/code/mypkg/sub1      0.001s
=== RUN   TestSub
--- FAIL: TestSub (0.00s)
FAIL
FAIL    _/home/baiju/code/mypkg/sub2      0.002s
\end{lstlisting}

As you can see above, the \texttt{TestHello} test was skipped as it
doesn't match "Sub" pattern.

The chapter on testing has more details about writing test cases.

\textit{golangci-lint}\index{lint} is a handy program to run various lint tools and
normalize their output. This program is useful to run through continuous
integration. You can download the program from here:
\url{https://github.com/golangci/golangci-lint}\\

The supported lint tools include Vet, Golint, Varcheck, Errcheck, Deadcode,
Gocyclo among others. \textit{golangci-lint} allows to enable/disable the lint
tools through a configuration file.

\section{Formatting Code}

Go has a recommended source code formatting\index{tool!format}.  To
format any Go source file to conform to that format, it's just a
matter of running one command.  Normally you can integrate this
command with your text editor or IDE.  But if you really want to
invoke this program from command line, this is how you do it:

\begin{lstlisting}[numbers=none]
go fmt myprogram.go
\end{lstlisting}

In the above command, the source file name is explicitly specified.
You can also give package name:

\begin{lstlisting}[numbers=none]
go fmt github.com/baijum/introduction
\end{lstlisting}

The command will format source files and write it back to the same
file. Also it will list the files that is formatted.

\section{Generating Code}

If you have use case to generate Go code from a grammar, you may consider
the \textit{go generate}. In fact, you can add any command to be run before
compiling the code. You can add a special comment in your Go code with this
syntax:

\begin{lstlisting}[numbers=none]
//go:generate command arguments
\end{lstlisting}

For example, if you want to use \textit{peg}
(\url{https://github.com/pointlander/peg}), a Parsing Expression Grammar
implementation, you can add the command like this:

\begin{lstlisting}[numbers=none]
//go:generate peg -output=parser.peg.go grammar.peg
\end{lstlisting}

When you build the program, the parser will be generated and will be part of the
code that compiles.

\section{Embedding Code}

Go programs are normally distributed as a single binary. What if your program
need some files to run. Go has a feature to embed files in the binary. You can
embed any type of files, including text files and binary files. Some of the
commonly embedded files are SQL, HTML, CSS, JavaScript, and images. You can
embed individual files or diectories including nested sub-directories.

You need to import the \textit{embed} package and use the \texttt{//go:embed}
compiler directive to embed. Here is an example to embed an SQL file:

\begin{lstlisting}[numbers=none]
import _ "embed"

//go:embed database-schema.sql
var dbSchema string
\end{lstlisting}

As you can see, the "embed" package is imported with a blank identifier as it is
not directly used in the code. This is required to initialize the package to
embed files. The variable must be at package level and not at function or method
level.

The variable could be slice of bytes. This is useful for binary files. Here is
an example:

\begin{lstlisting}[numbers=none]
import _ "embed"

//go:embed logo.jpg
var logo []byte
\end{lstlisting}

If you need an entire directory, you can use the \texttt{embed.FS} as the type:

\begin{lstlisting}[numbers=none]
import "embed"

//go:embed static
var content embed.FS
\end{lstlisting}

\section{Displaying Documentation}

Go has good support for writing documentation along with source
code\index{tool!doc}.  You can write documentation for packages,
functions and custom defined types.  The Go tool can be used to
display those documentation.

To see the documentation for the current packages, run this command:

\begin{lstlisting}[numbers=none]
go doc
\end{lstlisting}

To see documentation for a specific package:

\begin{lstlisting}[numbers=none]
go doc strings
\end{lstlisting}

The above command shows documentation for the "strings" package.

\begin{lstlisting}[numbers=none]
go doc strings
\end{lstlisting}

If you want to see documentation for a particular function within that
package:

\begin{lstlisting}[numbers=none]
go doc strings.ToLower
\end{lstlisting}

or a type:

\begin{lstlisting}[numbers=none]
go doc strings.Reader
\end{lstlisting}

Or a method:

\begin{lstlisting}[numbers=none]
go doc strings.Reader.Size
\end{lstlisting}

\section{Find Suspicious Code Using Vet}

There is a handy tool named \textit{vet}\index{tool!vet} to find
suspicious code in your program.  Your program might compile and
run. But some of the results may not be desired output.

Consider this program:

\lstinputlisting[caption=Suspicious code]{code/tooling/suspicious/suspicious.go}

If you compile and run it.  It's going to be give some output.  But if
you observe the code, there is an unnecessary \texttt{\%s} format
string.

If you run \texttt{vet} command, you can see the issue:

%% TODO: Change the font size back to \small and change filename to
%% suspicious.go (Probably this can be changed after moving the
%% content to next page}

\begin{lstlisting}[numbers=none]
$ go vet susp.go
# command-line-arguments
./susp.go:9: Printf format %s reads arg #2,
but call has only 1 arg
\end{lstlisting}

Note: The \textit{vet} command is automatically run along with
the \textit{test} command.

\section{Exercises}

\textbf{Exercise 1:} Create a program with function to return "Hello, world!" and
write test and run it.

\textit{hello.go}:

\begin{lstlisting}[numbers=none]
package hello

// SayHello returns a "Hello word!" message
func SayHello() string {
     return "Hello, world!"
}
\end{lstlisting}

\textit{hello\_test.go}:

\begin{lstlisting}[numbers=none]
package hello

import "testing"

func TestSayHello(t *testing.T) {
     out := SayHello()
     if out != "Hello, world!" {
        t.Error("Incorrect message", out)
     }
}
\end{lstlisting}

To run the test:

\begin{lstlisting}[numbers=none]
go test . -v
\end{lstlisting}

\subsection{Additional Exercises}

Answers to these additional exercises are given in the Appendix A.

{\bfseries Problem 1:} Write a program with exported type and methods
with documentation strings.  Then print the documentation
using the \texttt{go doc} command.

\section*{Summary}

This chapter introduced the Go tool.  All the Go commands were
explained in detail.  Practical example usage was also given for each
command.  The chapter coverd how to build and run programs, running
tests, formatting code, and displaying documentation.  It also touched
upon few other handy tools.
